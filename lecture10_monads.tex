% !TeX document-id = {0c096819-60cf-4d9a-a18e-64c3d748af84}
% !TeX TXS-program:compile = txs:///pdflatex/[-shell-escape]

\documentclass[11pt]{beamer}
\usepackage[utf8]{inputenc}
\usepackage[T1,T2A]{fontenc}
\usepackage[russian]{babel}
\usepackage{color}
\usepackage{calc}
\usepackage{graphicx}
\usepackage{epstopdf}
\usepackage{hyperref}
\hypersetup{unicode,colorlinks}
\usetheme[progressbar=head,numbering=fraction,block=fill]{metropolis}
\usepackage{minted}
\usepackage{dejavu}
%\usepackage{adjustbox}  % Позволяет сузить куски кода (или текст) ровно настолько, чтобы уместиться в слайд
\usepackage{csquotes}
\usepackage{upquote}

\usemintedstyle{solarized-light}
\newminted[haskell]{haskell}{
    escapeinside=!!,
    mathescape=true,
    texcomments=true,
    beameroverlays=true,
    autogobble=true,
    fontsize=\small,
    breaklines=false  % Лучше сам поставлю переносы на удобных местах
}
\newminted[haskellsmall]{haskell}{
    escapeinside=!!,
    mathescape=true,
    texcomments=true,
    beameroverlays=true,
    autogobble=true,
    fontsize=\footnotesize,
    breaklines=false
}
\newminted[haskelltiny]{haskell}{
    escapeinside=!!,
    mathescape=true,
    texcomments=true,
    beameroverlays=true,
    autogobble=true,
    fontsize=\scriptsize,
    breaklines=false
}
\newmintinline[haskinline]{haskell}{
    escapeinside=!!,
    mathescape=true,
    beameroverlays=true,
    breaklines=true
}
\newminted[ghci]{text}{
    autogobble=true,
    fontsize=\small,
    breaklines=false
}
\newminted[ghcismall]{text}{
    autogobble=true,
    fontsize=\footnotesize,
    breaklines=false
}
\newminted[ghcitiny]{text}{
    autogobble=true,
    fontsize=\scriptsize,
    breaklines=false
}
\newmintinline[ghcinline]{text}{
    breaklines=true
}

\newcommand{\hackage}[1]{\href{https://hackage.haskell.org/package/#1}{#1}}

\vfuzz=20pt  % позволяет тексту дойти до номера слайда

\author{Алексей Романов}
\subtitle{Функциональное программирование на Haskell}
%\logo{}
\institute{МИЭТ}
\subject{Функциональное программирование на Haskell}
%\setbeamercovered{transparent}
%\setbeamertemplate{navigation symbols}{}


\title{Лекция 10: монады}

\begin{document}
\begin{frame}[plain]
  \maketitle
\end{frame}

\begin{frame}[fragile]
  \frametitle{Монады}
  \begin{itemize}
    \item Монады расширяют возможности аппликативных функторов так же, как те расширяют возможности функторов.
    \item А именно, там добавляется
          \begin{haskell}
            class Applicative m => Monad m where
              (>>=) :: m a -> (a -> m b) -> m b
              (>>) :: m a -> m b -> m b
              mx >> my =!\pause! mx >>=!\pause! \_ -> my
          \end{haskell}
          \pause
    \item По историческим причинам есть ещё \haskinline|return = pure|.
          \pause
    \item Больше похоже на знакомые типы, если поменять аргументы местами:
          \begin{haskell}
            fmap  ::   (a -> b) -> f a -> f b
            (<*>) :: f (a -> b) -> f a -> f b
            (=<<) :: (a -> m b) -> m a -> m b
          \end{haskell}
  \end{itemize}
\end{frame}

\begin{frame}[fragile]
  \frametitle{Что можно сделать с монадами}
  \begin{itemize}
    \item В прошлой лекции:
          \begin{displayquote}
            Структура результата аппликативного вычисления зависит только от структуры аргументов, а не от значений внутри них
          \end{displayquote}
    \item Монады снимают это ограничение. Теперь можно написать
          \begin{haskell}
            ifM :: Monad m => m Bool -> m a -> m a -> m a
            ifM mCond mThen mElse = mCond >>=!\pause! \cond ->!\pause!
              if cond then mThen else mElse
          \end{haskell}
          и проверить:
          \begin{haskell}
            ghci> ifM [True] [1] [2,3]
            [1]
            ghci> ifM [False] [1] [2,3]
            [2,3]
          \end{haskell}
  \end{itemize}
\end{frame}


\begin{frame}[fragile]
  \frametitle{Стрелки Клейсли}
  \begin{itemize}
    \item Тип второго аргумента \haskinline|(>>=)|, \haskinline|a -> m b|, где \haskinline|m| "--- монада, называется \emph{стрелкой Клейсли}.
    \item \haskinline|a| и \haskinline|b| могут быть как переменными типа, так и конкретными типами.
    \item Стрелки Клейсли можно композировать:
          \begin{haskellsmall}
            (>=>) :: Monad m => (a -> m b) -> (b -> m c) ->
                                (a -> m c)
            f >=> g = \x ->!\pause! f x >>=!\pause! g
          \end{haskellsmall}
    \item Есть и аналоги \haskinline|filter|, \haskinline|map| и т.д., работающие со стрелками Клейсли. Их можно найти в \href{https://hackage.haskell.org/package/base-4.18.0.0/docs/Control-Monad.html}{документации \haskinline{Control.Monad}}.
  \end{itemize}
\end{frame}

\begin{frame}[fragile]
  \frametitle{Законы монад}
  \begin{itemize}
    \item \haskinline|pure x >>= f| $\equiv$ \haskinline|f x|
    \item \haskinline|mx >>= pure| $\equiv$ \haskinline|mx|
    \item \haskinline|(mx >>= f) >>= g| $\equiv$ \haskinline|mx >>= (\x -> f x >>= g)|
    \item Может быть, они не совсем интуитивны.
          \pause
    \item С помощью монадической композиции они записываются естественнее:
    \item \haskinline|pure >=> f| $\equiv$ \haskinline|f|
    \item \haskinline|f >=> pure| $\equiv$ \haskinline|f|
    \item \haskinline|(f >=> g) >=> h| $\equiv$ \haskinline|f >=> (g >=> h)| \pause
    \item Кроме этого, должно быть согласование с \haskinline|<*>|: \\
          \haskinline|mf <*> mx| $\equiv$ \haskinline|mf >>= (\f -> mx >>= (\x -> pure (f x)))|
          (тоже громоздко, но удобная запись немного позже)
  \end{itemize}
\end{frame}

\begin{frame}[fragile]
  \frametitle{Примеры монад}
  \begin{itemize}[<+->]
    \item \haskinline|Maybe| "--- монада. Определим:
          \begin{haskell}
            instance Monad Maybe where
              -- (>{}>=) ::\pause Maybe a -> (a -> Maybe b) -> Maybe b
              Nothing >>= _ =!\pause! Nothing
              Just x  >>= f =!\pause! f x
          \end{haskell}
    \item Списки тоже монада:
          \begin{haskell}
            -- (>{}>=) ::\pause [a] -> (a -> [b]) -> [b]
            []     >>= _ =!\pause! []
            (x:xs) >>= f =!\pause! f x ++ (xs >>= f)
          \end{haskell}
          Или проще:
          \begin{haskell}
            xs >>= f = [y |!\pause! x <- xs, y <- f x]
          \end{haskell}
  \end{itemize}
\end{frame}

\begin{frame}[fragile]
  \frametitle{Ещё примеры монад}
  \begin{itemize}
    \item \haskinline|instance Monad Identity|, где \haskinline|newtype Identity a = Identity a|.
    \item \haskinline|instance Monad (Either c)|.
    \item \haskinline|instance Monad ((->) c)|: функции с фиксированным типом аргумента.
    \item \haskinline|instance Monoid c => Monad ((,) c)|: пары с фиксированным типом первого элемента, если этот тип "--- моноид.
  \end{itemize}
\end{frame}

\begin{frame}[fragile]
  \frametitle{\ldots и не монад}
  \begin{itemize}
    \item Есть и примеры аппликативных функторов, для которых нельзя определить экземпляр монады:
    \item
          \begin{haskell}
            newtype ConstInt a = ConstInt Int
            fmap f (ConstInt x) = ConstInt x
            pure _ = 0
            ConstInt x <*> ConstInt y = ConstInt (x + y)
          \end{haskell}
    \item Легко увидеть, что \haskinline|pure x >>= f| $\equiv$ \haskinline|f x| не может выполняться ни для какого определения \haskinline|>>=|: правая часть зависит от \haskinline|x|, а левая нет.
          \pause
    \item Ещё пример "--- \haskinline|ZipList| (если не допускать \emph{только} бесконечные списки, как в домашнем задании, или активное использование $\bot$).
  \end{itemize}
\end{frame}

\begin{frame}[fragile]
  \frametitle{\haskinline|do|-нотация}
  \begin{itemize}
    \item Для монад есть специальный синтаксис, которым часто удобнее пользоваться.
    \item Скажем, у нас есть цепочка операций
          \begin{haskellsmall}
            action1 >>= (\x1 -> action2 x1 >>=
              (\x2 -> action3 x1 x2 >> action4 x1))
          \end{haskellsmall}
    \item
          Сначала перепишем так:
          \begin{haskellsmall}
            action1 >>= \x1 ->
              action2 x1 >>= \x2 ->
              action3 x1 x2 >>
              action4 x1
          \end{haskellsmall}
    \item
          В \haskinline|do|-блоке строки вида \haskinline|action1 >>= \x1 ->| превращаются в \haskinline|x1 <- action1|, а \haskinline|>>| пропадает:
          \begin{haskellsmall}
            do x1 <- action1
               x2 <- action2 x1
               action3 x1 x2
               action4 x1
          \end{haskellsmall}
  \end{itemize}
\end{frame}

\begin{frame}[fragile]
  \frametitle{Законы монад в \haskinline|do|-нотации}
  \begin{itemize}[<+->]
    \item Законы монад также можно записать через \haskinline|do|:
    \item
          \begin{haskell}
            do y <- pure x    !$\mathtt{\equiv}$!    f x
               f y
          \end{haskell}
    \item
          \begin{haskell}
            do x <- mx        !$\mathtt{\equiv}$!    mx
               pure x
          \end{haskell}
    \item
          \begin{haskell}
            do x <- mx        !$\mathtt{\equiv}$!    do y <- do x <- mx
               y <- f x       !\textcolor{white}{$\mathtt{\equiv}$}!               f x
               g y            !\textcolor{white}{$\mathtt{\equiv}$}!               g y
          \end{haskell}
    \item
          \begin{haskell}
            mf <*> mx         !$\mathtt{\equiv}$!    do f <- mf
                              !\textcolor{white}{$\mathtt{\equiv}$}!       x <- mx
                              !\textcolor{white}{$\mathtt{\equiv}$}!       pure (f x)
          \end{haskell}
  \end{itemize}
\end{frame}

\begin{frame}[fragile]
  \frametitle{Общая форма \haskinline|do|-нотации}
  \begin{itemize}
    \item Каждая строка \haskinline|do|-блока имеет вид \haskinline|образец <- м_выражение|, \haskinline|let образец = выражение| или просто \haskinline|м_выражение|.
    \item Первые два вида не могут быть в конце.
    \item \haskinline|м_выражение| должно иметь тип \haskinline|m a| для какой-то монады \haskinline|m| и типа \haskinline|a|.
    \item \haskinline|m| одна для всех строк, \haskinline|a| могут различаться.
    \item \haskinline|образец| в строке с \haskinline|<-| имеет тип \haskinline|a|.
    \item Если \haskinline|m| "--- экземпляр \haskinline|MonadFail|, то \haskinline|образец| может не быть обязательным для всех значений типа \haskinline|a|, например \haskinline|Just x <- pure Nothing|.
    \item Подробности в документации \haskinline|MonadFail|, но у нас в курсе такой необходимости нет.
  \end{itemize}
\end{frame}

\begin{frame}[fragile]
  \frametitle{Функции над произвольными монадами}
  \begin{itemize}[<+->]
    \item Кроме уже виденных \haskinline|=<<| и \haskinline|>=>|, в \haskinline|Prelude| и \haskinline|Control.Monad| есть ещё функции, которые работают для любых монад (или аппликативов).
    \item \haskinline|join :: Monad m => m (m a) -> m a|. Эту функцию можно было бы взять как базовую и выразить \haskinline|>>=| через неё.
    \item \haskinline|sequence :: Monad m => [m a] -> m [a]|. На самом деле, в библиотеке более общий вариант.
    \item \haskinline[fontsize=\small]|mapM :: Monad m => (a -> m b) -> [a] -> m [b]|
    \item
          \begin{haskell}
    zipWithM :: Applicative m =>
      (a -> b -> m c) -> [a] -> [b] -> m [c]
    \end{haskell}
    \item Подставьте конкретные монады (например, \haskinline|Maybe| и \haskinline|[]|) и подумайте, что функции будут делать для них.
  \end{itemize}
\end{frame}

\begin{frame}[fragile]
  \frametitle{Монада \haskinline|State|}
  \begin{itemize}
    \item Рассмотрим ещё один, более сложный пример:
    \item
          \begin{haskellsmall}
            newtype State s a = State { runState :: s -> (a, s) }
          \end{haskellsmall}
          Это \enquote{вычисления с состоянием}, которые могут выдать результат, зависящий от состояния и изменить это состояние.
    \item Как сделать их аппликативным функтором?
          \begin{haskellsmall}
            instance Functor (State s) where
              fmap f (State gx) = State (\s1 ->!\pause!
                let (x, s2) = gx s1
                in!\pause! (f x, s2))

            instance Applicative (State s) where
              pure x = State (\s1 ->!\pause! (x, s1))
              (State gf) <*> (State gx) = State (\s1 ->
                let (f, s2) = gf s1!\pause!
                (x, s3) = gx s2
                in!\pause! (f x, s3))
          \end{haskellsmall}
  \end{itemize}
\end{frame}

\begin{frame}[fragile]
  \frametitle{Монада \haskinline|State|}
  \begin{itemize}
    \item А теперь монадой:
          \begin{haskellsmall}
            instance Monad (State s) where
              State gx >>= f = State (\s1 ->
                let (x, s2) = gx s1!\pause!
                State gy = f x
                in!\pause! gy s2)
          \end{haskellsmall}
          \pause
    \item Определим вспомогательные функции:
          \begin{haskellsmall}
            get :: State s s
            get = State!\pause! (\s -> (s, s))

            put :: s -> State s ()
            put x = State!\pause! (\_ -> ((), x))

            modify :: (s -> s) -> State s ()
            modify f = do x <- get
                          put (f x)
          \end{haskellsmall}
  \end{itemize}
\end{frame}

\begin{frame}[fragile]
  \frametitle{Дополнительное чтение}
  \begin{itemize}
    \item \href{https://www.snoyman.com/blog/2017/01/functors-applicatives-and-monads}{Functors, Applicatives, and Monads}
    \item \href{https://wiki.haskell.org/Typeclassopedia}{Typeclassopedia} (ещё раз)
    \item \href{http://dev.stephendiehl.com/hask/#monads}{What I Wish I Knew When Learning Haskell: Monads}
    \item \href{https://wiki.haskell.org/All_About_Monads}{All About Monads}
    \item И две более сложные монады:
          \begin{itemize}
            \item \href{http://blog.sigfpe.com/2008/12/mother-of-all-monads.html}{The Mother of all Monads}
            \item \href{https://lukepalmer.wordpress.com/2008/08/10/mindfuck-the-reverse-state-monad/}{Mindfuck: The Reverse State Monad}
            \item \href{https://tech-blog.capital-match.com/posts/5-the-reverse-state-monad.html}{The Curious Time-Traveling Reverse State Monad}
          \end{itemize}
  \end{itemize}
\end{frame}

\end{document}
