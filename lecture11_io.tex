% !TeX document-id = {40e7f911-37dd-4856-a048-b5b803f1756a}
% !TeX TXS-program:compile = txs:///pdflatex/[-shell-escape]

\documentclass[11pt]{beamer}
\usepackage[utf8]{inputenc}
\usepackage[T1,T2A]{fontenc}
\usepackage[russian]{babel}
\usepackage{color}
\usepackage{calc}
\usepackage{graphicx}
\usepackage{epstopdf}
\usepackage{hyperref}
\hypersetup{unicode,colorlinks}
\usetheme[progressbar=head,numbering=fraction,block=fill]{metropolis}
\usepackage{minted}
\usepackage{dejavu}
%\usepackage{adjustbox}  % Позволяет сузить куски кода (или текст) ровно настолько, чтобы уместиться в слайд
\usepackage{csquotes}
\usepackage{upquote}

\usemintedstyle{solarized-light}
\newminted[haskell]{haskell}{
    escapeinside=!!,
    mathescape=true,
    texcomments=true,
    beameroverlays=true,
    autogobble=true,
    fontsize=\small,
    breaklines=false  % Лучше сам поставлю переносы на удобных местах
}
\newminted[haskellsmall]{haskell}{
    escapeinside=!!,
    mathescape=true,
    texcomments=true,
    beameroverlays=true,
    autogobble=true,
    fontsize=\footnotesize,
    breaklines=false
}
\newminted[haskelltiny]{haskell}{
    escapeinside=!!,
    mathescape=true,
    texcomments=true,
    beameroverlays=true,
    autogobble=true,
    fontsize=\scriptsize,
    breaklines=false
}
\newmintinline[haskinline]{haskell}{
    escapeinside=!!,
    mathescape=true,
    beameroverlays=true,
    breaklines=true
}
\newminted[ghci]{text}{
    autogobble=true,
    fontsize=\small,
    breaklines=false
}
\newminted[ghcismall]{text}{
    autogobble=true,
    fontsize=\footnotesize,
    breaklines=false
}
\newminted[ghcitiny]{text}{
    autogobble=true,
    fontsize=\scriptsize,
    breaklines=false
}
\newmintinline[ghcinline]{text}{
    breaklines=true
}

\newcommand{\hackage}[1]{\href{https://hackage.haskell.org/package/#1}{#1}}

\vfuzz=20pt  % позволяет тексту дойти до номера слайда

\author{Алексей Романов}
\subtitle{Функциональное программирование на Haskell}
%\logo{}
\institute{МИЭТ}
\subject{Функциональное программирование на Haskell}
%\setbeamercovered{transparent}
%\setbeamertemplate{navigation symbols}{}


\title{Лекция 11: монада IO и взаимодействие с внешним миром}

\begin{document}

\begin{frame}[plain]
  \maketitle
\end{frame}

\begin{frame}[fragile]
  \frametitle{Монада IO}
  \begin{itemize}
    \item Среди многих монад стандартной библиотеки, \haskinline|IO| играет особую роль.
    \item Значение типа \haskinline|IO a| описывает вычисление с результатом типа \lstinline|a| и побочными эффектами.
    \item В первую очередь имеются в виду ввод-вывод, как видно из названия, но также изменяемые переменные и т.д.
  \end{itemize}
\end{frame}

\begin{frame}[fragile]
  \frametitle{Программы командной строки}
  \begin{itemize}
    \item Исполняемая программа в Haskell задаётся значением
          \haskinline|main :: IO a| (почти всегда \haskinline|IO ()|).
    \item Это аналог \mintinline{c}|int main(int argc, char** argv)| в C, \mintinline{java}|public static void main(char[] args)| в Java и т.д.
    \item Соответственно, есть доступ к аргументам командной строки и возможность указать код выхода:
    \item \haskinline|System.Environment.getArgs :: IO [String]|. Там же есть функции для чтения переменных окружения.
    \item \haskinline|System.Exit.exitWith :: ExitCode -> IO a|.
  \end{itemize}
\end{frame}

\begin{frame}[fragile]
  \frametitle{Интерактивные программы}
  \begin{itemize}
    \item Из консоли можно читать текст, вводимый пользователем:
          \begin{haskell}
    getLine ::`\pause` IO String
    \end{haskell}
    \item И писать в неё:
          \begin{haskell}
    putStr, putStrLn ::`\pause` String -> IO ()
    print :: Show a => a -> IO () 
    \end{haskell}
          \pause
    \item Простой пример: читаем любую строку, выводим её длину и повторяем (а на \haskinline|"quit"| выходим):
          \pause
          \begin{haskell}
    main = do
      line <- getLine
      if line != "quit"
        then do
          print (length line)
          main
        else
          pure ()  -- или exitSuccess
    \end{haskell}
  \end{itemize}
\end{frame}

\begin{frame}[fragile]
  \frametitle{Работа с файлами}
  \begin{itemize}
    \item Модуль \href{http://hackage.haskell.org/package/base-4.12.0.0/docs/System-IO.html}{\haskinline|System.IO|}.
    \item Напомню, \haskinline|type FilePath = String|.
    \item Простейшие операции это чтение и запись в файл:
          \begin{haskell}
    readFile :: FilePath -> IO String
    writeFile, appendFile :: 
      FilePath -> String -> IO ()
    \end{haskell}
    \item
          Или можно открыть файл один раз, получить \haskinline|Handle| (дескриптор файла) и работать с ним:
          \begin{haskell}
    openFile, openBinaryFile :: 
      FilePath -> IOMode -> IO Handle 
    hPutStrLn :: Handle -> String -> IO ()
    hGetStr...
    \end{haskell}
          \pause
    \item
          Функции работы с консолью сводятся к этим:
          \begin{haskell}
    getLine = hGetLine stdin
    \end{haskell}
  \end{itemize}
\end{frame}

\begin{frame}[fragile]
  \frametitle{Работа с файлами (2)}
  \begin{itemize}
    \item В чём проблема с кодом вроде
          \begin{haskell}
    do
      file <- openFile path ReadWriteMode
      ...
      hClose file
    \end{haskell}
          \pause
    \item Что, если на каком-то из шагов случится исключение? Мы остаёмся с открытым файлом.
    \item Если это случится много раз, программа вылетит.
          \pause
    \item Чтобы закрыть файл и при исключении:
          \begin{haskell}
    withFile path ReadWriteMode $ \file -> do
      ...
    \end{haskell}
    \item Это частный случай функции \haskinline|bracket|.
  \end{itemize}
\end{frame}

\begin{frame}[fragile]
  \frametitle{Ленивый ввод-вывод}
  \begin{itemize}
    \item Функция \haskinline|hGetContents| не читает содержимое файла, а сразу возвращает строку, по мере доступа к которой читается файл.
    \item С одной стороны, это хорошо: не нужно явно указывать, сколько читать.
    \item Но если закрыть файл до того, как он реально прочитан, строка закончится как если бы она дошла до конца файла:
          \begin{haskell*}{fontsize=\footnotesize}
            wrong = do
            fileData <- withFile "test.txt" ReadMode hGetContents
            putStr fileData
          \end{haskell*}
    \item Нужно использовать данные внутри \haskinline|withFile|:
          \begin{haskell*}{fontsize=\footnotesize}
            right = withFile "test.txt" ReadMode $ \file -> do
            fileData <- hGetContents file
            putStr fileData
          \end{haskell*}
  \end{itemize}
\end{frame}

\begin{frame}[fragile]
  \frametitle{\haskinline|ByteString| и \haskinline|Text|}
  \begin{itemize}
    \item Мы знаем, что \haskinline|String| занимает очень много места в памяти, что ухудшает и время работы.
    \item Вместо него есть \haskinline|Data.ByteString.ByteString|, по сути представляющий собой массив байтов, и \haskinline|Data.ByteString.Lazy.ByteString| как ленивый список таких массивов.
    \item Многие функции для работы с ними называются так же, как для \haskinline|String|, но живут в соответствующих модулях (поэтому используется \haskinline|import qualified|).
    \item Они годятся для бинарных данных (или ASCII, с помощью \haskinline|Data.ByteString[.Lazy].Char8|).
    \item Для текста аналогичный пакет \haskinline|text| и типы в модулях \haskinline|Data.Text[.Lazy]|.
  \end{itemize}
\end{frame}

\begin{frame}[fragile]
  \frametitle{Случайные значения}
  \begin{itemize}
    \item Модуль \haskinline|System.Random| содержит класс
    \item
          \begin{haskell*}{fontsize=\footnotesize}
            class RandomGen g where
            next :: g -> (Int, g)
            split :: g -> (g, g)
            genRange :: g -> (Int, Int)
            genRange _ = (minBound, maxBound)
          \end{haskell*}
          описывающий генераторы случайных \haskinline|Int|.
    \item Есть тип \haskinline|StdGen| и \haskinline|instance RandomGen StdGen|.
    \item Глобальный генератор живёт в \haskinline|IORef|:
          \begin{haskell*}{fontsize=\footnotesize}
            getStdGen :: IO StdGen
            setStdGen :: StdGen -> IO ()
            newStdGen :: IO StdGen  -- применяет split к getStdGen
            getStdRandom :: (StdGen -> (a, StdGen)) -> IO a
          \end{haskell*}
    \item Типы \haskinline|next| и \haskinline|getStdRandom| могут напомнить про \haskinline|State| и не зря!
    \item Реализуем \haskinline|newStdGen| и \haskinline|getStdRandom|.
  \end{itemize}
\end{frame}

\begin{frame}[fragile]
  \frametitle{Случайные значения (ответ)}
  \begin{itemize}
    \item
          \begin{haskell}
    newStdGen = do 
      currGen <- getStdGen
      let (newGen1, newGen2) = split currGen
      setStdGen newGen2
      pure newGen1
    \end{haskell}
          \pause
    \item
          \begin{haskell}
    getStdRandom f = do 
      currGen <- getStdGen
      (result, newGen) = f currGen
      setStdGen newGen
      pure result
    \end{haskell}
          \pause
    \item Можем выразить одно через другое?
          \pause
          \begin{haskell}
    newStdGen = getStdRandom split
    \end{haskell}
  \end{itemize}
\end{frame}

\begin{frame}[fragile]
  \frametitle{Случайные значения (2)}
  \begin{itemize}
    \item Ещё один класс в \haskinline|System.Random| описывает типы, для которых можно получить случайные значения, если есть ГСЧ:
    \item
          \begin{haskell}
    class Random a where 
      randomR :: RandomGen g => (a, a) -> g -> (a, g) 
      random :: RandomGen g => g -> (a, g) 
    \end{haskell}
          Первый аргумент \haskinline|randomR|: диапазон, в котором берутся значения.
    \item Опять видим тип, похожий на \haskinline|State|.
    \item
          \begin{haskell}
    randomRIO :: Random a => (a, a) -> IO a
    \end{haskell}
          использует глобальный генератор.
  \end{itemize}
\end{frame}

\begin{frame}[fragile]
  \frametitle{Побег из \haskinline|IO|}
  \begin{itemize}
    \item Если у нас есть значение типа \haskinline|IO a|, можно ли превратить его просто в \haskinline|a|?
          \pause
    \item На самом деле есть функция \haskinline|unsafePerformIO :: IO a -> a|.
    \item Но само название говорит о её небезопасности.
    \item Конкретное использование может быть и безопасным, но для этого надо знать довольно много о внутренностях Haskell.
          \pause
    \item Поэтому в рамках этого курса мы её использовать не будем, как и другие \haskinline|unsafe*IO|.
    \item А есть ещё \haskinline|accursedUnutterablePerformIO|\ldots
  \end{itemize}
\end{frame}

\begin{frame}[fragile]
  \frametitle{Дополнительное чтение}
  \begin{itemize}
    \item TODO
  \end{itemize}
\end{frame}


\end{document}