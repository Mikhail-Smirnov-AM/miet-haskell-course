\documentclass[10pt]{beamer}
\usepackage[utf8]{inputenc}
\usepackage[T1,T2A]{fontenc}
\usepackage[russian]{babel}
\usepackage{color}
\usepackage{calc}
\usepackage{graphicx}
\usepackage{epstopdf}
\usepackage{hyperref}
\hypersetup{unicode,colorlinks}
\usetheme[progressbar=head,numbering=fraction,block=fill]{metropolis}
\usepackage{minted}
\usepackage{dejavu}
%\usepackage{adjustbox}  % Позволяет сузить куски кода (или текст) ровно настолько, чтобы уместиться в слайд
\usepackage{csquotes}
\usepackage{upquote}

\usemintedstyle{solarized-light}
\newminted[haskell]{haskell}{
    escapeinside=!!,
    mathescape=true,
    texcomments=true,
    beameroverlays=true,
    autogobble=true,
    fontsize=\small,
    breaklines=false  % Лучше сам поставлю переносы на удобных местах
}
\newminted[haskellsmall]{haskell}{
    escapeinside=!!,
    mathescape=true,
    texcomments=true,
    beameroverlays=true,
    autogobble=true,
    fontsize=\footnotesize,
    breaklines=false
}
\newminted[haskelltiny]{haskell}{
    escapeinside=!!,
    mathescape=true,
    texcomments=true,
    beameroverlays=true,
    autogobble=true,
    fontsize=\scriptsize,
    breaklines=false
}
\newmintinline[haskinline]{haskell}{
    escapeinside=!!,
    mathescape=true,
    beameroverlays=true,
    breaklines=true
}
\newminted[ghci]{text}{
    autogobble=true,
    fontsize=\small,
    breaklines=false
}
\newminted[ghcismall]{text}{
    autogobble=true,
    fontsize=\footnotesize,
    breaklines=false
}
\newminted[ghcitiny]{text}{
    autogobble=true,
    fontsize=\scriptsize,
    breaklines=false
}
\newmintinline[ghcinline]{text}{
    breaklines=true
}

\newcommand{\hackage}[1]{\href{https://hackage.haskell.org/package/#1}{#1}}

\vfuzz=20pt  % позволяет тексту дойти до номера слайда

\author{Алексей Романов}
\subtitle{Функциональное программирование на Haskell}
%\logo{}
\institute{МИЭТ}
\subject{Функциональное программирование на Haskell}
%\setbeamercovered{transparent}
%\setbeamertemplate{navigation symbols}{}


\title{Лекция 5: функции как значения}

\begin{document}
\begin{frame}[plain]
  \maketitle
\end{frame}

\begin{frame}[fragile]
  \frametitle{Функции как значения}
  \begin{itemize}
    \item Как упоминалось в начале курса, одно из оснований ФП состоит в том, что функции могут использоваться как значения.
    \item В Haskell можно выразиться сильнее:\pause
    \item Функции это и есть просто значения, тип которых имеет форму \haskinline|ТипПараметра -> ТипРезультата| для каких-то \haskinline|ТипПараметра| и \haskinline|ТипРезультата|.\pause
    \item Мы уже видели примеры этого в равноправии функций и \only<4>{\textbf{других} }переменных.
  \end{itemize}
\end{frame}

\begin{frame}[fragile]
  \frametitle{Функции высших порядков}
  \begin{itemize}
    \item В частности, функции могут принимать на вход функции.
    \item То есть тип параметра сам может быть функциональным типом.
    \item Тривиальный пример:
          \begin{haskell}
            foo :: (Char -> Bool) -> Bool
            foo f = f 'a'

            ghci> foo isLetter !\pause!
            True
          \end{haskell}
    \item Скобки вокруг типа параметра здесь необходимы.\pause
    \item Функции, параметры которых "--- функции, называются \emph{функциями высших порядков (ФВП)}.\pause
    \item Часто ими также считают функции, возвращающие функции, но не в Haskell (скоро увидим почему).
  \end{itemize}
\end{frame}

\begin{frame}[fragile]
  \frametitle{Лямбда-выражения}
  \begin{itemize}
    \item В \haskinline|foo| с прошлого слайда можем также передать свою новую функцию, определив её локально:
          \begin{haskell}
            foo isCyrillic where 
                isCyrillic c = let lc = toLower c
                              in 'а' <= lc && lc <= 'я'
          \end{haskell}
\pause
    \item Но имя этой функции на самом деле не нужно.
    \item Как и вообще функциям, которые создаются только как аргументы для других (или как результаты).\pause
    \item Вместо этого зададим её через \emph{лямбда-выражение}
          \begin{haskell}
            foo (\c -> let lc = toLower c 
                      in 'а' <= lc && lc <= 'я')
          \end{haskell}
          \pause
    \item[]
    \item Кстати, почему это определение неверно?
          \pause
    \item Как ни странно, в Юникоде \haskinline|'ё' > 'я'|.
    \item И не все буквы кириллицы используются в русском.
  \end{itemize}
\end{frame}

\begin{frame}[fragile]
  \frametitle{Лямбда-выражения}
  \begin{itemize}
    \item Вообще, два определения
          \begin{haskell}
            функция образец = результат

            функция = \образец -> результат
          \end{haskell}
          эквивалентны.\pause
    \item Одно исключение: \href{https://wiki.haskell.org/Monomorphism_restriction}{для второго может быть выведен менее общий тип}.\pause
    \item Лямбда-выражение для функции с несколькими параметрами пишется
          \begin{haskell}
            \образец1 ... образецN -> результат
          \end{haskell}
          \pause
    \item Например,
          \begin{haskell}
            Data.List.sortBy (\x y -> compare y x) list
          \end{haskell}
    \item Что делает это выражение?\pause
    \item Сортирует список по убыванию.
  \end{itemize}
\end{frame}

\begin{frame}[fragile]
  \frametitle{Лямбда-выражения и \haskinline|case|}
  \begin{itemize}
    \item Если в обычном определении функции несколько уравнений, например
          \begin{haskell}
            not True  = False
            not False = True
          \end{haskell}
          то в лямбда-выражении придётся использовать \haskinline|case|:\pause
          \begin{haskell}
            not = \x -> case x of
                True  -> False
                False -> True
          \end{haskell}
          \pause
          или с расширением \haskinline|LambdaCase|
          \begin{haskell}
            not = \case
                True  -> False
                False -> True
          \end{haskell}
  \end{itemize}
\end{frame}

\begin{frame}[fragile]
  \frametitle{Несколько стандартных ФВП}
  \begin{itemize}
    \item В \haskinline|Prelude| есть три функции второго порядка, которые очень часто встречаются в коде Haskell:
          \begin{haskell}
            ($) :: (a -> b) -> a -> b
            f $ x =!\pause! f x!\pause!

            (.) :: (b -> c) -> (a -> b) -> (a -> c)
            g . f =!\pause! \x ->!\pause! g (f x)

            flip :: (a -> b -> c) -> (b -> a -> c)
            flip f = !\pause! \y x ->!\pause! f x y
          \end{haskell}
    \item И ещё две в \haskinline|Data.Function|:
          \begin{haskell}
            (&) = flip (.)
            (&) ::!\pause! (a -> b) -> (b -> c) -> (a -> c)

            on :: (b -> b -> c) -> (a -> b) -> 
                (a -> a -> c)
          \end{haskell}
  \end{itemize}
\end{frame}

\begin{frame}[fragile]
  \frametitle{Избавление от скобок с помощью \haskinline|$| и \haskinline|.|}
  \begin{itemize}
    \item Казалось бы, в чём смысл \haskinline|$|: зачем писать \haskinline|f $ x| вместо \haskinline|f x|?\pause
    \item Этот оператор имеет минимальный возможный приоритет, так что \haskinline|f (x + y)| можно записать как \haskinline|f $ x + y|.\pause
    \item И он правоассоциативен, так что \haskinline|f (g (h x))| можно записать как \haskinline|f $ g $ h x|.\pause
    \item Но более принято \haskinline|f . g . h $ x|.\pause
    \item \haskinline|.| тоже правоассоциативен, но уже с максимальным приоритетом.\pause
    \item[]
    \item Пока, наверное, проще читать и писать код со скобками, но стандартный стиль Haskell предпочитает их избегать.
    \item Только не перестарайтесь!
  \end{itemize}
\end{frame}

\begin{frame}[fragile]
  \frametitle{Правда о функциях многих переменных}
  \begin{itemize}
    \item Настала пора раскрыть тайну функций многих переменных в Haskell:\pause
    \item Их не существует.\pause
    \item \haskinline|->| "--- правоассоциативный оператор, так что \haskinline|Тип1 -> Тип2 -> Тип3| это на самом деле \haskinline|Тип1 -> (Тип2 -> Тип3)|: функция, возвращающая функцию.\pause
    \item \haskinline|\x y -> результат| это сокращение для \haskinline|\x -> \y -> результат|, а со скобками \haskinline|\x -> (\y -> результат)|.\pause
    \item Применение функций (не \haskinline|$|, а пробел), наоборот, левоассоциативно. То есть \haskinline|f x y| читается как \haskinline|(f x) y|.
  \end{itemize}
\end{frame}

\begin{frame}[fragile]
  \frametitle{Каррирование}
  \begin{itemize}
    \item Обычно в математике для сведения функций нескольких аргументов к функциям одного аргумента используется декартово произведение: $A \times B \to C$, а не $A \to (B \to C)$.\pause
    \item В Haskell тоже можно было бы писать
          \begin{haskell}
            foo :: (Int, Int) -> Int
            foo (x, y) = x + y
          \end{haskell}
          для определения функций и \haskinline|foo (1, 2)| для вызова.\pause
    \item Но Haskell здесь следует традиции $\lambda$-исчисления.\pause
    \item Эти подходы эквивалентны, так как множества $A \times B \to C$ и $A \to (B \to C)$ всегда изоморфны \pause ($C^{A \cdot B} = C^{B^A}$).
    \item В Haskell этот изоморфизм реализуют функции\pause
          \begin{haskell}
            curry :: ((a, b) -> c) -> a -> b -> c
            uncurry :: (a -> b -> c) -> (a, b) -> c
          \end{haskell}
  \end{itemize}
\end{frame}

\begin{frame}[fragile]
  \frametitle{Частичное применение}
  \begin{itemize}
    \item Преимущество каррированных функций в том, что естественным образом появляется частичное применение.
    \item То есть мы можем применить функцию \enquote{двух аргументов} только к первому и останется функция одного аргумента:
          \begin{haskell}
            ghci> :t sortBy
            sortBy :: (a -> a -> Ordering) -> [a] -> [a]
            ghci> :t sortBy (flip compare) !\pause!
            sortBy (flip compare) :: Ord a => [a] -> [a]
          \end{haskell}
          \pause
    \item Заметьте, что здесь частичное применение в двух местах: \pause\haskinline|flip| можно теперь рассматривать как функцию трёх аргументов!
  \end{itemize}
\end{frame}

\begin{frame}[fragile]
  \frametitle{Сечения операторов}
  \begin{itemize}
    \item Для применения бинарного оператора только к первому аргументу можно использовать его префиксную форму:
          \begin{haskell}
            ghci> :t (+) 1
            (+) 1 :: Num a => a -> a
          \end{haskell}
          \pause
    \item А ко второму? Можно использовать лямбду
          \begin{haskell}
            \x -> x / 2
          \end{haskell}
          или \haskinline|flip|
          \begin{haskell}
            flip (/) 2
          \end{haskell}
          \pause
    \item Но есть специальный синтаксис \haskinline|(арг оп)| и \haskinline|(оп арг)|:
          \begin{haskell}
            ghci> (1 `div`) 2 !\pause!
            0
            ghci> (/ 2) 4 !\pause!
            2.0
          \end{haskell}
  \end{itemize}
\end{frame}

\begin{frame}[fragile]
  \frametitle{Сечения кортежей}
  \begin{itemize}
    \item Расширение \haskinline|TupleSections| позволяет частично применять конструкторы кортежей:
          \begin{haskell}
            ghci> :set -XTupleSections
            ghci> (, "I", , "Love", True) 1 False !\pause!
            (1,"I",False,"Love",True)
            ghci> :t (, "I", , "Love", True) !\pause!
            (, "I", , "Love", True)
              :: t1 -> t2 -> (t1, [Char], t2, [Char], Bool)
          \end{haskell}
  \end{itemize}
\end{frame}

\begin{frame}[fragile]
  \frametitle{$\eta$-эквивалентность (сокращение аргументов)}
  \begin{itemize}
    \item Рассмотрим два определения
          \begin{haskell}
            foo' x = foo x
            -- или foo' = \x -> foo x 
          \end{haskell}
    \item \haskinline|foo' y == foo y|, какое бы \haskinline|y| (и \haskinline|foo|) мы не взяли.\pause
    \item Поэтому мы можем упростить определение:
          \begin{haskell}
            foo' = foo
          \end{haskell}
\pause
    \item Это также относится к случаям, когда совпадают часть аргументов в конце:
          \begin{haskell}
            foo' x y z w = foo (y + x) z w
          \end{haskell}
          эквивалентно
          \begin{haskell}
            foo' x y = foo (y + x)
          \end{haskell}
  \end{itemize}
\end{frame}

\begin{frame}[fragile]
  \frametitle{$\eta$-эквивалентность (сокращение аргументов)}
  \begin{itemize}
    \item И когда само определение такой формы не имеет, но может быть к ней преобразовано
          \begin{haskell}
            root4 x = sqrt (sqrt x)
          \end{haskell}
          можно переписать как
          \begin{haskell}
            root4 x = (sqrt . sqrt) x

            root4 = sqrt . sqrt
          \end{haskell}
    \item Ещё пример:
          \begin{haskell}
            ($) :: (a -> b) -> a -> b
            f $ x = f x

            ($) f x = f x !\pause!
            ($) f = f !\pause!
            ($) f = id f !\pause!
            ($) = id
          \end{haskell}
  \end{itemize}
\end{frame}

\begin{frame}[fragile]
  \frametitle{Бесточечный стиль}
  \begin{itemize}
    \item Как видим, с помощью композиции и других операций можно дать определение некоторых функций без использования переменных.
    \item Это называется \emph{бесточечным стилем} (переменные рассматриваются как точки в пространстве значений).\pause
    \item Оказывается, что очень многие выражения в Haskell имеют бесточечный эквивалент.
    \item На \href{https://wiki.haskell.org/Pointfree}{Haskell Wiki} можно найти инструменты, позволяющие переводить между стилями.
    \item Опять же, не перестарайтесь:
          \begin{haskell}
            > pl \x y -> compare (f x) (f y)
            ((. f) . compare .)
          \end{haskell}
          \pause
    \item В языках семейств Forth и APL бесточечный стиль является основным.
  \end{itemize}
\end{frame}

\begin{frame}[fragile]
  \frametitle{Программирование, направляемое типами}
  \begin{itemize}
    \item При реализации полиморфных функций очень часто их тип подсказывает реализацию.
    \item Пример:
          \begin{haskell}
            foo :: (a -> b, b -> Int) -> (a -> Int)
          \end{haskell}
          Какую часть реализации можно написать сразу?
          \begin{haskell}
            foo !\pause!(f, g) = !\pause!\x ->!\pause! ?
          \end{haskell}
          Типы переменных:\pause
          \begin{haskell}
            f :: a -> b, g :: b -> Int, !\pause!x :: a, ? :: Int
          \end{haskell}
          Можно увидеть, что \haskinline|?| может быть \haskinline|g ??|, где \haskinline|?? :: |\pause\haskinline|b = |\pause\haskinline|f x|.
  \end{itemize}
\end{frame}

\begin{frame}[fragile]
  \frametitle{Типизированные дыры в коде}
  \begin{itemize}
    \item Эти рассуждения не обязательно делать вручную.
    \item Если в выражении (а не в образце) использовать дыру \haskinline|_| (или \haskinline|_название|), то увидим ожидаемый в этом месте тип.
    \item
          \begin{haskell}
            foo :: (a -> b, b -> Int) -> (a -> Int)
            foo (f, g) = \x -> g _
          \end{haskell}
          выдаст сообщение
          \begin{ghci}
            Found hole `_' with type: b
            Where: `b' is a rigid type variable bound by
            the type signature for foo :: (a -> b, b -> Int) -> (a -> Int)
            Relevant bindings include ...
          \end{ghci}
    \item Дыр может быть несколько.
  \end{itemize}
\end{frame}

\end{document}
