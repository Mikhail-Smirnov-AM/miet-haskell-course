\documentclass[10pt]{beamer}
\usepackage[utf8]{inputenc}
\usepackage[T1,T2A]{fontenc}
\usepackage[russian]{babel}
\usepackage{color}
\usepackage{calc}
\usepackage{graphicx}
\usepackage{epstopdf}
\usepackage{hyperref}
\hypersetup{unicode,colorlinks}
\usetheme[progressbar=head,numbering=fraction,block=fill]{metropolis}
\usepackage{minted}
\usepackage{dejavu}
%\usepackage{adjustbox}  % Позволяет сузить куски кода (или текст) ровно настолько, чтобы уместиться в слайд
\usepackage{csquotes}
\usepackage{upquote}

\usemintedstyle{solarized-light}
\newminted[haskell]{haskell}{
    escapeinside=!!,
    mathescape=true,
    texcomments=true,
    beameroverlays=true,
    autogobble=true,
    fontsize=\small,
    breaklines=false  % Лучше сам поставлю переносы на удобных местах
}
\newminted[haskellsmall]{haskell}{
    escapeinside=!!,
    mathescape=true,
    texcomments=true,
    beameroverlays=true,
    autogobble=true,
    fontsize=\footnotesize,
    breaklines=false
}
\newminted[haskelltiny]{haskell}{
    escapeinside=!!,
    mathescape=true,
    texcomments=true,
    beameroverlays=true,
    autogobble=true,
    fontsize=\scriptsize,
    breaklines=false
}
\newmintinline[haskinline]{haskell}{
    escapeinside=!!,
    mathescape=true,
    beameroverlays=true,
    breaklines=true
}
\newminted[ghci]{text}{
    autogobble=true,
    fontsize=\small,
    breaklines=false
}
\newminted[ghcismall]{text}{
    autogobble=true,
    fontsize=\footnotesize,
    breaklines=false
}
\newminted[ghcitiny]{text}{
    autogobble=true,
    fontsize=\scriptsize,
    breaklines=false
}
\newmintinline[ghcinline]{text}{
    breaklines=true
}

\newcommand{\hackage}[1]{\href{https://hackage.haskell.org/package/#1}{#1}}

\vfuzz=20pt  % позволяет тексту дойти до номера слайда

\author{Алексей Романов}
\subtitle{Функциональное программирование на Haskell}
%\logo{}
\institute{МИЭТ}
\subject{Функциональное программирование на Haskell}
%\setbeamercovered{transparent}
%\setbeamertemplate{navigation symbols}{}


\title{Лекция 4: классы типов}
\date{\today}

\begin{document}
\begin{frame}[plain]
\maketitle
\end{frame}

\begin{frame}[fragile]
\frametitle{Классы типов}
\begin{itemize}
    \item Полиморфные функции, которые мы видели в прошлый раз, имеют одно определение для любых параметров типов.
    \item Часто нужно определить функции по-разному для разных типов.
    \item Для этого используются классы типов.
\begin{lstlisting}[basicstyle=\ttfamily\small]
class Eq a where
    (==), (/=) :: a -> a -> Bool
    x == y = not (x /= y)
    x /= y = not (x == y)
\end{lstlisting}
    \item \lstinline|Eq| "--- название класса.
    \item \lstinline|a| "--- тип, относящийся к этому классу.
    \item \lstinline|(==)| и \lstinline|(/=)| "--- функции-члены класса, с определениями по умолчанию.
    \item Их нужно определить для типа \lstinline|a| \pause(достаточно одну).
\end{itemize}
\end{frame}

\begin{frame}[fragile]
\frametitle{Экземпляры классов}
\begin{itemize}
    \item Чтобы объявить, что тип \lstinline|Bool| относится к классу \lstinline|Eq|:
\begin{lstlisting}[basicstyle=\ttfamily\small]
instance Eq Bool where
    -- (==) :: Bool -> Bool -> Bool
    True == True = True
    False == False = True !\pause!
    _ == _ = False
\end{lstlisting}
    \item При этом \lstinline|a| в сигнатуре членов заменяется на тип, для которого определяется экземпляр.
    \item Ещё пример:\pause
\begin{lstlisting}[basicstyle=\ttfamily\small]
instance Eq IpAddress where
    -- (==) :: IpAddress -> IpAddress -> Bool
    Ip4Address x == Ip4Address y = x == y
    Ip6Address x == Ip6Address y = x == y
    _ == _ = False
\end{lstlisting}
    \only<3>{\item Рекурсивно ли это определение?}
    \item<4-> Это не рекурсия: \lstinline|==| в правой части относится к другому типу (\lstinline|String|).
\end{itemize}
\end{frame}

\begin{frame}[fragile]
\frametitle{Ограниченный полиморфизм}
\begin{itemize}
    \item Если спросить GHCi про тип \lstinline|==|, получим
\begin{lstlisting}[basicstyle=\ttfamily\small]
Prelude> :t (==)
(==) :: Eq a => a -> a -> Bool
\end{lstlisting}
    \item Читается \enquote{\lstinline|a -> a -> Bool| для любого \lstinline|a| из \lstinline|Eq|}.
    \item Часть перед \lstinline|=>| называется контекстом. 
    \item Его элементы "--- ограничения. Когда их больше одного, они пишутся в скобках через запятую.\pause
    \item Типы полиморфных функций, \emph{использующих} \lstinline|==| и \lstinline|/=|, пишутся аналогично. Определим:
\begin{lstlisting}[basicstyle=\ttfamily\small]
foo [] = True
foo [x] = True
foo (x : tail@(y : _)) = x == y && foo tail
\end{lstlisting}
    Какой у неё тип (и смысл)? \pause
    \item \lstinline[basicstyle=\ttfamily\small]|foo :: Eq a => [a] -> Bool| \pause
    \item Функция проверяет, равны ли все элементы списка.
\end{itemize}
\end{frame}

\begin{frame}[fragile]
\frametitle{Экземпляры для полиморфных типов}
\begin{itemize}
    \item Попробуем определить \lstinline|Eq| для \lstinline|Maybe a|:\pause
\begin{lstlisting}[basicstyle=\ttfamily\small]
instance Eq (Maybe a) where
    -- (==) :: Maybe a -> Maybe a -> Bool
    Just x == Just y = x == y
    Nothing == Nothing = True
    _ == _ = False
\end{lstlisting}
    \only<2>{\item Верное ли это определение?} \pause
    \item Здесь \lstinline|a| не обязательно \lstinline|Eq|, и поэтому \lstinline|x == y| \\не скомпилируется.
    \item А можно ли так?
\begin{lstlisting}[basicstyle=\ttfamily\small]
instance Eq (Maybe a) where
    (==) :: Eq a => Maybe a -> Maybe a -> Bool
    Just x == Just y = x == y
    Nothing == Nothing = True
    _ == _ = False
\end{lstlisting} \pause
    \item Нет: эта сигнатура для \lstinline|(==)| более ограничена, чем заданная классом.
\end{itemize}
\end{frame}

\begin{frame}[fragile]
\frametitle{Экземпляры для полиморфных типов \\с ограничениями}
\begin{itemize}
    \item Правильно так:
\begin{lstlisting}
instance Eq a => Eq (Maybe a) where
    Just x == Just y = x == y
    Nothing == Nothing = True
    _ == _ = False
\end{lstlisting} 
    \item Читаем \enquote{\lstinline|Maybe a| относится к \lstinline|Eq| только тогда, когда \lstinline|a| относится к \lstinline|Eq|}.
    \item Контексты экземпляров подчиняются тем же правилам, что контексты функций.
    \item Ограничения могут упоминать другие классы, а не только тот, экземпляр которого определяется.
    \item Не может быть много экземпляров одного класса для одного типа с разными условиями.
\end{itemize}
\end{frame}

\begin{frame}[fragile]
\frametitle{Аналоги в других языках}
\begin{itemize}
    \item Как перегрузка функций (или операторов), но глубже встроенная в систему типов.
    \item Перегруженные функции "--- просто разные функции, связанные только общим именем.
    \item Здесь обязательно близкие типы и общий смысл.
    \item И функции могут быть использованы другими полиморфными функциями. \pause
    \item Не путайте классы типов и классы в ООП, общее только название!
    \item В C++ есть близкое понятие концепций. \pause
    \item До C++20 только для документации\pause, а теперь их можно объявлять и использовать в коде. \pause
    \item Часто сравнивают с интерфейсами в Java/C\#, но это подходит хуже: \pause в классах типов есть \enquote{статические} и бинарные методы, совсем другой смысл функции, возвращающей такой тип.
\end{itemize}
\end{frame}

\begin{frame}[fragile]
\frametitle{Другие важные классы}
\begin{itemize}
\item 
Линейно упорядоченные типы:
\begin{lstlisting}
class Eq a => Ord a where
    compare :: a -> a -> Ordering
    (<), (<=), (>), (>=) :: a -> a -> Bool 
    max, min :: a -> a -> a
\end{lstlisting}
    \begin{itemize}
    \item \lstinline|Eq a| "--- надкласс \lstinline|Ord|, то есть любой экземпляр \lstinline|Ord| должен быть и экземпляром \lstinline|Eq|. \pause
    \item Логически стрелка скорее в другом направлении:\\ \lstinline|a| относится к \lstinline|Ord| $\implies$ \lstinline|a| относится к \lstinline|Eq|.
    \end{itemize}
\pause
\item 
Типы, ограниченные сверху и снизу:
\begin{lstlisting}
class Bounded a where
    minBound, maxBound :: a 
\end{lstlisting}
    \begin{itemize}
    \item Это не подкласс \lstinline|Ord| потому, что порядок, для которого эти значения "--- границы, может быть частичным.
    \end{itemize}
\end{itemize}
\end{frame}

\begin{frame}[fragile]
\frametitle{Преобразования в строки и обратно}
\begin{itemize}
    \item 
    Типы, значения которых можно превратить в строки:
\begin{lstlisting}[basicstyle=\ttfamily\small]
class Show a where
    show :: a -> String 
    ...
\end{lstlisting}\pause
\item Или прочитать из строк:
\begin{lstlisting}[basicstyle=\ttfamily\small]
class Read a where ...

read :: Read a => String -> a
\end{lstlisting}\pause
    \item Для нас то, что за многоточиями, неважно, но подробности описаны \href{http://hackage.haskell.org/package/base-4.10.1.0/docs/Prelude.html#g:22}{в документации}.\pause
    \item \lstinline|read| должна быть обратной к \lstinline|show|.
    \item Заметьте полиморфизм \lstinline|read| по возвращаемому типу:
\begin{lstlisting}[basicstyle=\ttfamily\small]
Prelude> read "2" :: Integer
2
Prelude> read "2" :: Double
2.0
\end{lstlisting}
\end{itemize}
\end{frame}

\begin{frame}[fragile]
\frametitle{Опасность \lstinline|read|}
\begin{itemize}
    \item Что случится, если передать \lstinline|read| неподходящий аргумент?
\begin{lstlisting}
Prelude> read "1" :: [Bool] !\pause!
*** Exception: Prelude.read: no parse
\end{lstlisting}
    \item Если вы не знаете \emph{точно}, что текст содержит значение, используйте безопасную
\begin{lstlisting}
readMaybe :: Read a => String -> Maybe a
\end{lstlisting}
    из модуля \lstinline|Text.Read|.
    \pause
    \item[]
    \item Это не единственная функция в Prelude с такой проблемой.
    \item Пакет \href{https://hackage.haskell.org/package/safe}{safe} содержит безопасные варианты для всех таких функций (посмотрите его документацию).
    \end{itemize}
\end{frame}

\begin{frame}[fragile]
\frametitle{Класс перечислимых типов}
\begin{itemize}
    \item 
    \lstinline|Enum| содержит типы, значения которых образуют последовательность.
\begin{lstlisting}
class Enum a where
    succ, pred :: a -> a
    toEnum   :: Int -> a
    fromEnum :: a -> Int
    enumFrom :: a -> [a] -- и ещё 3 вариации
\end{lstlisting}
    \pause
    \item Для него есть специальный синтаксис: 
    \begin{itemize}
        \item \lstinline|[n..]| означает \lstinline|enumFrom n|;
        \item \lstinline|[n..m]| "--- \lstinline|enumFromTo n m|;
        \item \lstinline|[n,m..]| "--- \lstinline|enumFromThen n m|;
        \item \lstinline|[n,m..l]| "--- \lstinline|enumFromThenTo n m l|.
    \end{itemize}
    \pause
    \item Для \lstinline|Float| и \lstinline|Double| есть экземпляры \lstinline|Enum|, так что можно писать \lstinline|[1.0..]|\only<4->{, но они довольно странные}: 
\begin{lstlisting}
Prelude> [1.2..2.0] !\pause!
[1.2,2.2]
\end{lstlisting}
\end{itemize}
\end{frame}

\begin{frame}[fragile]
\frametitle{Иерархия числовых классов}
\begin{itemize}
    \item Вернёмся к \enquote{числовым типам} из первой лекции.
    \item Это тоже классы типов, образующие довольно сложную иерархию.\pause
\begin{lstlisting}
class Num a where
    (+), (-), (*), negate, abs, signum, fromInteger

class (Num a, Ord a) => Real a where
    toRational
    
class (Real a, Enum a) => Integral a where
    quot, rem, div, mod, quotRem, divMod
    toInteger
    
class Num a => Fractional a where
    (/), recip, fromRational    
\end{lstlisting}
    
\end{itemize}
\end{frame}

\begin{frame}[fragile]
\frametitle{Иерархия числовых классов: дробные типы}
\begin{itemize}
\begin{lstlisting}[basicstyle=\ttfamily\small]
class Fractional a => Floating a where 
    exp, log, sqrt, (**), logBase, 
    pi, sin, cos, ...

class (Real a, Fractional a) => RealFrac a where 
    properFraction :: Integral b => a -> (b, a)
    truncate, round, ceiling, floor :: 
        Integral b => a -> b

class (RealFrac a, Floating a) => RealFloat a where
    isNaN, isInfinite :: a -> Bool
    ...
    
fromIntegral :: (Integral a, Num b) => a -> b
realToFrac :: (Real a, Fractional b) => a -> b
\end{lstlisting}
\item \href{http://hackage.haskell.org/package/base-4.10.1.0/docs/Prelude.html#g:6}{Полная документация.}
\end{itemize}
\end{frame}

\begin{frame}[fragile]
\frametitle{Числовые литералы}
\begin{itemize}
    \item Литерал \lstinline|0| означает \lstinline|fromInteger (0 :: Integer)| и соответственно получает тип \pause\lstinline|Num a => a|.
    \item Аналогично для дробных: \lstinline|1.1| представляется как рациональное и превращается в \lstinline|fromRational ((11 % 10) :: Rational)| типа \lstinline|Fractional a => a|.
    \item Здесь:
\begin{lstlisting}
module Data.Ratio where
data Ratio a -- экспортировано без конструктора
type Rational = Ratio Integer
(%) :: Integral a => a -> a -> Ratio a
\end{lstlisting}
    \pause
    \item Использование рациональных чисел для дробных литералов позволяет избежать ошибок округления.\pause
    \item Конечно, при преобразовании в \lstinline|Float| или \lstinline|Double| они вернутся.
\end{itemize}
\end{frame}

\begin{frame}[fragile]
\frametitle{Алгебраические классы}
\begin{itemize}
    \item Для разных структур из общей алгебры или линейной алгебры можно определить соответствующие им классы. В стандартной библиотеке есть
    \item моноиды в \lstinline|Prelude|:
\begin{lstlisting}
class Monoid a where
    mempty :: a
    mappend :: a -> a -> a
    mconcat :: [a] -> a
    
x <> y = mappend x y -- синоним для mappend
\end{lstlisting}
    \item и полугруппы в \href{https://hackage.haskell.org/package/base-4.10.1.0/docs/Data-Semigroup.html}{\lstinline|Data.Semigroup|}. \pause
    \item Ещё несколько из теории категорий, но это предмет будущих лекций.
\end{itemize}
\end{frame}

\begin{frame}[fragile]
\frametitle{Определение функций внутри или вне класса}
\begin{itemize}
    \item При определении функций над классом часто приходится выбирать из двух вариантов: 
    \begin{itemize}
        \item член класса с определением по умолчанию;
        \item функция, определённая вне класса.
    \end{itemize}
    \item Например, \lstinline|(/=)| можно было бы определить вне класса:
\begin{lstlisting}
(/=) :: Eq a => a -> a -> Bool
x /= y = not (x == y)
\end{lstlisting}\pause
    \item Такую функцию нельзя реализовать по-разному для разных экземпляров.\pause
    \item Если для каких-то типов есть реализация лучше той, что по умолчанию, лучше сделать функцию членом класса.
    \pause
    \begin{itemize}
        \item Очевидная проблема: вы можете не знать (или не подумать) о таких типах.
    \end{itemize}
    \pause
    \item Или если реализация по умолчанию не всегда верна \pause(но тогда почему это реализация по умолчанию?).
\end{itemize}
\end{frame}

% TODO Похоже, неверно: compare оптимально? Поискать пример лучше

%\begin{frame}[fragile]
%\frametitle{Определение функций внутри или вне класса:\\случай \lstinline|Ord|}
%\begin{itemize}
%    \item Ещё пример: в \lstinline|Ord| можно было бы оставить только \lstinline|(<=)| членом класса.
%    \item Или только \lstinline|compare|.
%    \item Но определение \lstinline|(<)| и других функций через них не всегда оптимально.
%    \item (Подумайте, почему!)
%\end{itemize}
%\end{frame}

\begin{frame}[fragile]
\frametitle{Законы классов типов}
\begin{itemize}
    \item Рассмотрим такой экземпляр типа:
\begin{lstlisting}
data Weird = Weird Int

instance Eq Weird where
    Weird x == Weird y = x /= y
\end{lstlisting}
    \item Даже не зная смысла \lstinline|Weird| мы видим, что что-то здесь не так. \only<1>{Что?}\pause
    \item Это определение нерефлексивно, то есть \lstinline|Weird 1 /= Weird 1|.
    \item У всех классов типов (почти) есть законы.
    \item То есть наличия функций типов, указанных в определении, недостаточно;
    \item для корректных экземпляров должны выполняться определённые законы.    
\end{itemize}
\end{frame}

\begin{frame}[fragile]
\frametitle{Законы классов типов}
\begin{itemize}
    \item Например, для \lstinline|Eq| законы такие:\pause
\begin{lstlisting}
!$\forall$! x :: a. x == x
!$\forall$! x, y :: a. x == y !$\implies$! y == x
!$\forall$! x, y, z :: a. x == y !$\wedge$! y == z !$\implies$! x == z !\pause!

!$\forall$! x, y :: a; Eq b; f :: a -> b. x == y !$\implies$\pause! 
    f x == f y
\end{lstlisting}\pause
    \item Законы могут говорить и о связи между классами. Например, если тип одновременно \lstinline|Ord| и \lstinline|Bounded|, то естественно требовать
\begin{lstlisting}
!$\forall$! x :: a. x <= maxBound
!$\forall$! x :: a. x >= minBound
\end{lstlisting}
    \pause
    \item Компилятор не может проверить законы.
    \item Поэтому ответственность за их соблюдение лежит на программисте.
\end{itemize}
\end{frame}

\begin{frame}[fragile]
\frametitle{Автоматический вывод экземпляров}
\begin{itemize}
    \item Почти все определения \lstinline|(==)| устроены одинаково:\pause
    \begin{itemize}
        \item Типы всех полей всех конструкторов должны быть \lstinline|Eq|;
        \item Значения с одинаковым конструктором равны, если равны все их поля (или полей нет);
        \item Значения с разными конструкторами не равны;
    \end{itemize}
    \item Почти, но не все: \pause для рациональных чисел не так.\pause
    \item Чтобы не писать такие определения каждый раз, после \lstinline|data| или \lstinline|newtype| ставят \lstinline|deriving (Eq)|.
    \item \lstinline|deriving| также работает для \lstinline|Ord|, \lstinline|Show|, \lstinline|Read|, \lstinline|Enum| и \lstinline|Bounded|.
    \item Порядок на значениях для \lstinline|Ord|, \lstinline|Enum| и \lstinline|Bounded| задаётся порядком конструкторов в определении.\pause
    \item Разные расширения GHC позволяют \lstinline|deriving| для других классов.
\end{itemize}
\end{frame}

\begin{frame}[fragile]
\frametitle{Классы типов и \lstinline|newtype|}
\begin{itemize}
    \item Для каждой пары класса и типа можно определить только один экземпляр.
    \item А что делать, если осмысленных экземпляров больше одного?
    \item Например, числа "--- моноиды по операциям \lstinline|+| и \lstinline|*|.\pause
    \item В таком случае мы можем завернуть существующий тип в \lstinline|newtype| и определить для нового:
\begin{lstlisting}[basicstyle=\ttfamily\footnotesize]
newtype Sum a = Sum { getSum :: a }
    deriving (Eq, Ord, Read, Show, Bounded, Num)

instance Num a => Monoid (Sum a) where
    mempty = Sum 0
    mappend (Sum x) (Sum y) = Sum (x + y) !\pause!

newtype Product a = Product { getProduct :: a }
    deriving (Eq, Ord, Read, Show, Bounded, Num)

instance Num a => Monoid (Product a) where ...
\end{lstlisting}
\end{itemize}
\end{frame}

\begin{frame}[fragile]
\frametitle{Классы типов и \lstinline|newtype|}
\begin{itemize}
    \item Если заметили, \lstinline|Num| не было в списке классов, для которых есть автоматический вывод экземпляров.
    \item Для \lstinline|newtype| можно вывести любой класс с помощью расширения \lstinline|GeneralizedNewtypeDeriving|.\pause
    \item Выведенное определение каждой функции вызывает ту же функцию завёрнутого типа с вставкой и разбором конструкторов где нужно:
\begin{lstlisting}
instance Num a => Num (Sum a) where
    Sum x + Sum y = Sum (x + y)
    Sum x * Sum y = Sum (x * y)
    negate (Sum x) = Sum (negate x)
    ...
\end{lstlisting}
\end{itemize}
\end{frame}

\begin{frame}[fragile]
\frametitle{Классы типов и модули}
\begin{itemize}
    \item \lstinline|Класс(..)| в списке импорта/экспорта означает класс со всеми методами, как для типа.
    \item Поскольку экземпляры не имеют имён, они экспортируются из модуля всегда.
    \item И импортируются всегда, когда модуль упоминается в списке импорта.\pause
    \item[]
    \item Из-за этого существование экземпляров для пары класс-тип не в одном модуле "--- проблема.
    \item Поэтому экземпляры обычно определяются либо в том модуле, где определён тип, либо в том, где определён класс.
    \item Остальные называются сиротами и по возможности их создания следует избегать.
    \item Это делается с помощью того же трюка с \lstinline|newtype|.
\end{itemize}
\end{frame}

\begin{frame}[fragile]
\frametitle{Ограничения классов типов в стандартном Haskell и расширения GHC}
\begin{itemize}
    \item В стандартном Haskell есть несколько ограничений на объявления классов типов и экземпляров.
    \item А в GHC расширений, которые их снимают. 
    \item Все перечислять не будем, они есть в  \href{http://downloads.haskell.org/~ghc/latest/docs/html/users_guide/glasgow_exts.html#class-declarations}{User's Guide, раздел 10.8}.
    \item Сейчас достаточно знать, что если что-то не компилируется из-за отсутствия \lstinline|-XFlexibleInstances|, \lstinline|-XFlexibleContexts| или \lstinline|-XInstanceSigs|, их спокойно можно включить.
    \item \lstinline|-XUndecidableInstances|, \lstinline|-XOverlappingInstances|, \lstinline|-XIncoherentInstances| опаснее, но в этом курсе в любом случае не понадобятся.
\end{itemize}
\end{frame}

%\begin{frame}[fragile]
%\frametitle{Классы с несколькими параметрами}
%\begin{itemize}
%    \item TODO
%\end{itemize}
%\end{frame}
%
%\begin{frame}[fragile]
%\frametitle{Типы как члены классов}
%\begin{itemize}
%    \item TODO
%\end{itemize}
%\end{frame}
    
\end{document}