\documentclass[10pt]{beamer}
\usepackage[utf8]{inputenc}
\usepackage[T1,T2A]{fontenc}
\usepackage[russian]{babel}
\usepackage{color}
\usepackage{calc}
\usepackage{graphicx}
\usepackage{epstopdf}
\usepackage{hyperref}
\hypersetup{unicode,colorlinks}
\usetheme[progressbar=head,numbering=fraction,block=fill]{metropolis}
\usepackage{minted}
\usepackage{dejavu}
%\usepackage{adjustbox}  % Позволяет сузить куски кода (или текст) ровно настолько, чтобы уместиться в слайд
\usepackage{csquotes}
\usepackage{upquote}

\usemintedstyle{solarized-light}
\newminted[haskell]{haskell}{
    escapeinside=!!,
    mathescape=true,
    texcomments=true,
    beameroverlays=true,
    autogobble=true,
    fontsize=\small,
    breaklines=false  % Лучше сам поставлю переносы на удобных местах
}
\newminted[haskellsmall]{haskell}{
    escapeinside=!!,
    mathescape=true,
    texcomments=true,
    beameroverlays=true,
    autogobble=true,
    fontsize=\footnotesize,
    breaklines=false
}
\newminted[haskelltiny]{haskell}{
    escapeinside=!!,
    mathescape=true,
    texcomments=true,
    beameroverlays=true,
    autogobble=true,
    fontsize=\scriptsize,
    breaklines=false
}
\newmintinline[haskinline]{haskell}{
    escapeinside=!!,
    mathescape=true,
    beameroverlays=true,
    breaklines=true
}
\newminted[ghci]{text}{
    autogobble=true,
    fontsize=\small,
    breaklines=false
}
\newminted[ghcismall]{text}{
    autogobble=true,
    fontsize=\footnotesize,
    breaklines=false
}
\newminted[ghcitiny]{text}{
    autogobble=true,
    fontsize=\scriptsize,
    breaklines=false
}
\newmintinline[ghcinline]{text}{
    breaklines=true
}

\newcommand{\hackage}[1]{\href{https://hackage.haskell.org/package/#1}{#1}}

\vfuzz=20pt  % позволяет тексту дойти до номера слайда

\author{Алексей Романов}
\subtitle{Функциональное программирование на Haskell}
%\logo{}
\institute{МИЭТ}
\subject{Функциональное программирование на Haskell}
%\setbeamercovered{transparent}
%\setbeamertemplate{navigation symbols}{}


\title{Лекция 2: основы синтаксиса (продолжение)}

\begin{document}
\begin{frame}[plain]
  \maketitle
\end{frame}

\begin{frame}[fragile]
  \frametitle{Существенные отступы}
  \begin{itemize}
    \item Вспомним виденное раньше
          \begin{haskell}
            let {x :: Integer; x = 2}
          \end{haskell}
          По правилам Haskell можно написать это же без фигурных скобок и точки с запятой:
          \begin{haskell}
            let x :: Integer
                x = 2
          \end{haskell}
    \item Если два выражения (или других фрагмента кода) относятся к одному уровню одного блока, то они должны иметь одинаковый отступ (начинаться в одном и том же столбце).
    \item Фрагмент кода, являющийся частью другого, должен иметь отступ больше той строки, где тот начинается.
    \item Но не обязательно больше начала его самого.
  \end{itemize}
\end{frame}
\begin{frame}[fragile]
  \frametitle{Правило преобразования отступов}
  \begin{itemize}
    \item Если после ключевых слов \haskinline|where|, \haskinline|let|, \haskinline|do|, \haskinline|of| нет открывающей фигурной скобки \haskinline|{|\footnote{Если она есть, то правило перестаёт действовать до соответствующей ей закрывающей.}, то такая скобка вставляется и начинается новый блок.
    \item Его базовый отступ "--- столбец, содержащий следующий непробельный символ.
    \item После этого каждая строка, которая начинается:
          \begin{itemize}
            \item в том же столбце: относится к этому блоку и перед ней ставится \haskinline|;|
            \item правее: продолжает предыдущую, перед ней ничего не ставится.
            \item левее: блок закончился, перед ней ставится \haskinline|}| (одновременно может закончиться несколько блоков).
          \end{itemize}
  \end{itemize}
\end{frame}

\begin{frame}[fragile]
  \frametitle{Условные выражения}
  \begin{itemize}
    \item В любом языке нужна возможность выдать результат в зависимости от какого-то условия.
    \item В Haskell это выражения \haskinline|if| и \haskinline|case|.
    \item Синтаксис \haskinline|if|:
          \begin{haskell}
            if условие then выражение1 else выражение2
          \end{haskell}
          Варианта без \haskinline|else| нет.
    \item Многострочно пишется так:
          \begin{haskell}
            if условие 
                then выражение1 
                else выражение2
          \end{haskell}
    \item Тип \haskinline|условия| обязательно \haskinline|Bool|.
    \item Типы \haskinline|выражения1| и \haskinline|выражения2| должны совпадать.
    \item Это же и тип всего  \haskinline|if ... then ... else ...|
    \item Ближе к \haskinline|?:|, чем к \haskinline|if| в C-подобных языках.
  \end{itemize}
\end{frame}

\begin{frame}[fragile]
  \frametitle{Сопоставление с образцом}
  \begin{itemize}
    \item Синтаксис \haskinline|case|: \\
          \begin{haskell}
            case выражение of 
                образец1 -> выражение1 
                образец2 -> выражение2
          \end{haskell}
    \item Образец это \enquote{форма} для значений типа, которая может содержать несвязанные переменные. Для конкретного значения он либо подходит (и связывает эти переменные), либо не подходит.
    \item Процедура вычисления:
          \begin{itemize}
            \item Вычислить значение \haskinline|выражения|.
            \item Сопоставить его с каждым образцом по очереди.
            \item Если первым подошёл \haskinline|образецN|, вычислить \haskinline|выражениеN| и вернуть его значение.
            \item Если ни один не подошёл, выкинуть ошибку.
          \end{itemize}
  \end{itemize}
\end{frame}

\begin{frame}[fragile]
  \frametitle{Образцы для известных нам типов}
  \begin{itemize}
    \item \haskinline|True| и \haskinline|False| "--- образцы для \haskinline|Bool|. Они подходят, только если значение совпадает с ними.
    \item Аналогично \haskinline|LT|, \haskinline|EQ| и \haskinline|GT| для \haskinline|Ordering|, а числовые литералы для числовых типов.
    \item Переменная "--- образец для любого типа, который подходит для любого значения.
          \begin{itemize}
            \item В Haskell все переменные в образцах \enquote{свежие} и перекрывают видимые снаружи переменные с тем же названием.
          \end{itemize}
    \item \haskinline|_| тоже подходит для любого значения любого типа и означает, что это значение не важно.
    \item Образцы похожи на выражения, но ими не являются: это новая синтаксическая категория!
  \end{itemize}
\end{frame}

\begin{frame}[fragile]
  \frametitle{Связь \haskinline|case| и \haskinline|if|}
  \begin{itemize}
    \item Пример:
          \begin{haskell}
            if условие 
                then выражение1 
                else выражение2
          \end{haskell}
          это ровно то же самое, что
          \begin{haskell}
            case условие of
                True -> выражение1 
                False -> выражение2
          \end{haskell}
          или
          \begin{haskell}
            case условие of
                True -> выражение1 
                _ -> выражение2
          \end{haskell}
  \end{itemize}
\end{frame}

\begin{frame}[fragile]
  \frametitle{Охраняющие условия}
  \begin{itemize}
    \item У каждого образца могут быть дополнительные условия (зависящие от его переменных):
          \begin{haskell}
            образец
                | условие1 -> выражение1 
                | условие2 -> выражение2
          \end{haskell}
          При удачном сопоставлении они проверяются по порядку. Если все оказались ложны, сопоставление переходит к следующему образцу.
    \item Последнее условие часто \haskinline|otherwise| (синоним \haskinline|True|), тогда хотя бы одно условие точно истинно.
    \item Чтобы сравнить переменную в образце с внешней, нужно использовать \haskinline|==| в условии:
          \begin{haskell}
            case foo x of
                y | y == x -> ... -- не то же, что x -> ...
                _ -> ...
          \end{haskell}
  \end{itemize}
\end{frame}

\begin{frame}[fragile]
  \frametitle{Многоветвенные \haskinline|if|}
    \begin{itemize}
      \item Цепочка \haskinline|if ... else if ...|, оформленная по правилам отступа, напоминает лестницу:
            \begin{haskell}
              if условие1
                  then результат1
                  else if условие2
                      then результат2
                      else результат3
            \end{haskell}
    \item С расширением \haskinline|MultiWayIf| можем написать:
          \begin{haskell}
            if | условие1 -> результат1
               | условие2 -> результат2
               | otherwise -> результат3
          \end{haskell}
    \item Это можно сделать и с \haskinline|case| без расширений: \only<1>{как?} \pause
          \begin{haskell}
            case () of _ | условие1 -> ...
          \end{haskell}
    \end{itemize}
\end{frame}

\begin{frame}[fragile]
  \frametitle{Определение функций по случаям}
  \begin{itemize}
    \item Часто тело функции "--- \haskinline|case| по аргументу, например.
          \begin{haskell}
            not x = case x of
                True -> False
                False -> True
          \end{haskell}
    \item Такие функции можно записать несколькими равенствами, по одному на ветвь \haskinline|case|:
          \begin{haskell}
            not True = False
            not False = True
          \end{haskell}
    \item Это работает и с охранными условиями:
          \begin{haskell}
            not x | x = False
                  | otherwise = True
          \end{haskell}
    \item и для нескольких параметров:
          \begin{haskell}
            nand True True = False
            nand _    _ = True
          \end{haskell}
  \end{itemize}
\end{frame}

\begin{frame}[fragile]
  \frametitle{Локальные определения: \haskinline|let|}
  \begin{itemize}
    \item Локальные определения дают две выгоды:
          \begin{itemize}
            \item Более читаемый код.
            \item Избавление от повторяющихся вычислений.
          \end{itemize}
    \item В Haskell два способа их задать: \haskinline|let| и \haskinline|where|.
    \item Синтаксис \haskinline|let|:
          \begin{haskell}
            let переменная1 = выражение1
                функция2 x = выражение2
                ...
            in выражение3
          \end{haskell}
    \item Всё \haskinline|let ... in ...| "--- выражение.
    \item Первое проявление ленивости: будут вычислены только те переменные, которые понадобятся для вычисления \haskinline|выражения3|.
  \end{itemize}
\end{frame}

\begin{frame}[fragile]
  \frametitle{Локальные определения: \haskinline|where|}
  \begin{itemize}
    \item Синтаксис \haskinline|where|:
          \begin{haskell}
            функция образец1 | условие1 = выражение3
                             | условие2 = выражение4
                where переменная1 = выражение1
                      функция2 x = выражение2
          \end{haskell}
    \item Видимы в условиях и в правых частях (но не для других образцов).
    \item Можно применить только к определениям.
    \item В том числе к локальным.
  \end{itemize}
\end{frame}

\begin{frame}[fragile]
  \frametitle{Модули}
  \begin{itemize}
    \item Программа на Haskell состоит из модулей.
    \item Модуль \haskinline|Модуль| (названия с заглавной буквы) определяется в файле \haskinline|Модуль.hs|:
          \begin{itemize}
            \item Для названия вида \haskinline|A.B.C| будет файл \ghcinline|A/B/C.hs|.
          \end{itemize}
          \begin{haskell}
            module Модуль(функция1) where

            import ...

            функция1 :: ...
            функция1 x = ...
          \end{haskell}
    \item \haskinline|функция1| экспортирована, она доступна другим модулям и GHCi. Все остальные "--- нет.
    \item Можно экспортировать всё, опустив список экспортов (включая скобки).
  \end{itemize}
\end{frame}

\begin{frame}[fragile]
  \frametitle{Импорт из другого модуля}
  \begin{itemize}
    \item У директивы \haskinline|import| есть много вариантов.
    \item По адресу \url{https://wiki.haskell.org/Import} есть полный перечень.
    \item Нам пока достаточно простейших:
    \item \begin{haskell}
            import Модуль(функция1, функция2)
          \end{haskell}
          импортирует конкретные функции.
    \item \begin{haskell}
            import Модуль
          \end{haskell}
          импортирует всё, что возможно.
    \item Импортированные функции доступны как \haskinline|функция| и как \haskinline|Модуль.функция|.
  \end{itemize}
\end{frame}

\begin{frame}[fragile]
  \frametitle{Загрузка модуля}
  \begin{itemize}
    \item В GHCi можно скомпилировать и загрузить модуль (вместе с зависимостями) командой \haskinline|:load Модуль|. Подробности в документации:
          \begin{itemize}
            \item \href{http://downloads.haskell.org/~ghc/latest/docs/html/users_guide/ghci.html#ghci-load-scope}{The effect of :load on what is in scope}
          \end{itemize}
    \item Под Windows можно просто дважды щёлкнуть на файл или нажать Enter в Проводнике.
    \item Потом при изменениях файла повторить \haskinline|:load| или сделать \haskinline|:reload|.
    \item Многие редакторы позволяют это автоматизировать.
  \end{itemize}
\end{frame}

\begin{frame}[fragile]
  \frametitle{Stack}
  \begin{itemize}
    \item Для сборки лабораторных мы используем \href{https://docs.haskellstack.org/en/stable/#quick-start-guide}{Stack}.
    \item Ещё одна распространённая система сборки: \href{https://cabal.readthedocs.io/}{Cabal}.
    \item Обе ставятся вместе с Haskell через \ghcinline|ghcup|.
    \item Основные команды можно посмотреть в документации, но нам пока достаточно
          \begin{itemize}
            \item \ghcinline|stack repl| для запуска GHCi с загруженными файлами проекта.
            \item \ghcinline|stack test| для запуска тестов.
          \end{itemize}
  \end{itemize}
\end{frame}

\end{document}
