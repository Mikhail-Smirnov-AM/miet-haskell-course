% !TeX document-id = {fc72019d-d3ba-4743-9e76-a08bcafbe161}
% !TeX TXS-program:compile = txs:///pdflatex/[-shell-escape]

\documentclass[11pt]{beamer}
\usepackage[utf8]{inputenc}
\usepackage[T1,T2A]{fontenc}
\usepackage[russian]{babel}
\usepackage{color}
\usepackage{calc}
\usepackage{graphicx}
\usepackage{epstopdf}
\usepackage{hyperref}
\hypersetup{unicode,colorlinks}
\usetheme[progressbar=head,numbering=fraction,block=fill]{metropolis}
\usepackage{minted}
\usepackage{dejavu}
%\usepackage{adjustbox}  % Позволяет сузить куски кода (или текст) ровно настолько, чтобы уместиться в слайд
\usepackage{csquotes}
\usepackage{upquote}

\usemintedstyle{solarized-light}
\newminted[haskell]{haskell}{
    escapeinside=!!,
    mathescape=true,
    texcomments=true,
    beameroverlays=true,
    autogobble=true,
    fontsize=\small,
    breaklines=false  % Лучше сам поставлю переносы на удобных местах
}
\newminted[haskellsmall]{haskell}{
    escapeinside=!!,
    mathescape=true,
    texcomments=true,
    beameroverlays=true,
    autogobble=true,
    fontsize=\footnotesize,
    breaklines=false
}
\newminted[haskelltiny]{haskell}{
    escapeinside=!!,
    mathescape=true,
    texcomments=true,
    beameroverlays=true,
    autogobble=true,
    fontsize=\scriptsize,
    breaklines=false
}
\newmintinline[haskinline]{haskell}{
    escapeinside=!!,
    mathescape=true,
    beameroverlays=true,
    breaklines=true
}
\newminted[ghci]{text}{
    autogobble=true,
    fontsize=\small,
    breaklines=false
}
\newminted[ghcismall]{text}{
    autogobble=true,
    fontsize=\footnotesize,
    breaklines=false
}
\newminted[ghcitiny]{text}{
    autogobble=true,
    fontsize=\scriptsize,
    breaklines=false
}
\newmintinline[ghcinline]{text}{
    breaklines=true
}

\newcommand{\hackage}[1]{\href{https://hackage.haskell.org/package/#1}{#1}}

\vfuzz=20pt  % позволяет тексту дойти до номера слайда

\author{Алексей Романов}
\subtitle{Функциональное программирование на Haskell}
%\logo{}
\institute{МИЭТ}
\subject{Функциональное программирование на Haskell}
%\setbeamercovered{transparent}
%\setbeamertemplate{navigation symbols}{}

\usepackage{syntax}

\title{Лекция 14: что дальше?}

\begin{document}
\begin{frame}[plain]
  \maketitle
\end{frame}

\begin{frame}[fragile]
  \frametitle{Нерассмотренные части Haskell: стандартные библиотеки}
  \begin{itemize}
    \item В самом начале упоминалось, что тип \haskinline{String} очень плох для практического использования. Альтернативы в пакетах \lstinline|text| и \lstinline|bytestring|.
    \item Коллекции:
          \begin{itemize}
            \item \lstinline|containers|: \haskinline|Data.Map|, \haskinline|Data.Set|, \haskinline|Data.Sequence|.
            \item \lstinline|unordered-containers|: \haskinline|Data.Hash{Map/Set}|.
            \item Массивы: \lstinline|array| и \lstinline|vector|.
          \end{itemize}
    \item Комбинаторы парсеров: \lstinline|megaparsec|.
    \item Сборка проектов и управление зависимостями: \lstinline|cabal + cabal-install| и \lstinline|stack|.
    \item Линзы (\lstinline|lens|): обобщение геттеров и сеттеров.
    \item Альтернативы \lstinline|show| для больших структур: \lstinline|pretty-show| и \lstinline|prettyprinter|.
  \end{itemize}
\end{frame}

\begin{frame}[fragile]
  \frametitle{Нерассмотренные части Haskell: язык}
  \begin{itemize}
    \item \href{https://downloads.haskell.org/~ghc/latest/docs/html/users_guide/glasgow_exts.html#type-families}{Семейства типов}.
    \item \href{https://downloads.haskell.org/~ghc/latest/docs/html/users_guide/glasgow_exts.html#arbitrary-rank-polymorphism}{Типы высших рангов} с \haskinline|forall| \emph{внутри} типа.
          \begin{itemize}
            \item Важный пример: монада \haskinline|ST|.
          \end{itemize}
    \item \href{https://downloads.haskell.org/~ghc/latest/docs/html/users_guide/glasgow_exts.html#existentially-quantified-data-constructors}{Экзистенциальные типы} (выраженные всё равно через \haskinline|forall|).
    \item \href{https://downloads.haskell.org/~ghc/latest/docs/html/users_guide/glasgow_exts.html#pattern-synonyms}{Синонимы образцов}.
    \item \href{https://downloads.haskell.org/~ghc/latest/docs/html/users_guide/glasgow_exts.html#template-haskell}{Template Haskell}: генерация кода во время компиляции.
          \begin{itemize}
            \item Могли видеть использование в документации QuickCheck/Hedgehog.
          \end{itemize}
    \item Исключения.
    \item Foreign Function Interface для интеграции с C.
  \end{itemize}
\end{frame}

\begin{frame}[fragile]
  \frametitle{Подходы к эффектам}
  \begin{itemize}
    \item Многие монады (и аппликативные функторы) можно рассматривать как побочные эффекты.
          \begin{itemize}
            \item \haskinline|IO|: понятно.
            \item \haskinline|State|: чтение и запись в память.
            \item \haskinline|[]|: недетерминированность
            \item \ldots
          \end{itemize}
          \pause
    \item Как их комбинировать?
    \item Композиция монад не обязательно монада.
          \pause
    \item Трансформеры монад: \lstinline|transformers| и \lstinline|mtl|
    \item Алгебраические эффекты
    \item Свободные (free) и \enquote{более свободные} (freer) монады
    \item Небольшой обзор: \href{https://ocharles.org.uk/posts/2016-01-26-transformers-free-monads-mtl-laws.html}{Monad transformers, free monads, mtl, laws and a new approach}
  \end{itemize}
\end{frame}

\begin{frame}[fragile]
  \frametitle{Функциональные структуры данных}
  \begin{itemize}
    \item Функциональные структуры данных очень отличаются от привычных!
    \item Классическая книга: Okasaki, Purely functional data structures
    \item \href{https://cstheory.stackexchange.com/questions/1539/whats-new-in-purely-functional-data-structures-since-okasaki}{Список \enquote{после Окасаки} (2010)}. И \href{https://www.reddit.com/r/haskell/comments/a4wk1i/seminal_works_after_purely_functional_data/ebm30ra/}{ещё немного}.
          \pause
    \item Функциональные структуры, позволяющие доступ к своим старым состояниям, полезны и в императивных языках.
          \begin{itemize}
            \item А изменяемые "--- в Haskell через \haskinline|IO| или \haskinline|ST|.
          \end{itemize}
  \end{itemize}
\end{frame}

\begin{frame}[fragile]
  \frametitle{Будущее Haskell}
  \begin{itemize}
    \item \href{https://www.tweag.io/posts/2017-03-13-linear-types.html}{Линейные типы}: значения, которые могут быть использованы только один раз.
          \begin{itemize}
            \item Меньше сборки мусора.
            \item Можно использовать изменяемые структуры данных.
            \item Безопасное изменение состояния типа (например, тип файла может включать, открыт ли он).
          \end{itemize}
          \pause
    \item \href{https://serokell.io/blog/2018/12/17/why-dependent-haskell}{Зависимые типы}: типы, параметризованные значениями.
          \begin{itemize}
            \item Стандартный пример: вектор с размером в типе.
            \item Библиотека \lstinline|singletons| позволяет эмулировать многие применения зависимых типов.
          \end{itemize}
          \pause
    \item Не совсем будущее: уточнённые типы (refinement types) в Liquid Haskell.
          \begin{itemize}
            \item \href{https://ruhaskell.org/posts/utils/2016/12/16/liquidhaskell-hello.html}{LiquidHaskell: знакомство} и \href{https://liquid.kosmikus.org/}{Liquid Haskell Tutorial}
          \end{itemize}
  \end{itemize}
\end{frame}

\begin{frame}[fragile]
  \frametitle{Альтернативы Haskell}
  \begin{itemize}
    \item OCaml: похож на Haskell без ленивости и (пока) классов типов.
    \item Erlang: динамический функциональный язык, ориентированный на многозадачность.
    \item Варианты Lisp (можно посмотреть Racket).
          \pause
    \item Для JVM: Scala, Clojure.
    \item Для .Net: F\# (почти тот же OCaml).
          \pause
    \item ФП без сборщика мусора: Rust.
          \begin{itemize}
            \item Мощная и интересная система типов, описывающая время жизни значений.
            \item Очень быстрое развитие.
            \item Реальная альтернатива C++.
          \end{itemize}
          \pause
    \item Языки с зависимыми типами: Agda, Idris, Coq.
          \pause
    \item Языки спецификаций: TLA+, Alloy.
  \end{itemize}
\end{frame}

\begin{frame}[fragile]
  \frametitle{Области математики, связанные с ФП}
  \begin{itemize}
    \item Лямбда-исчисление: то, с чего всё началось.
          \begin{itemize}
            \item И множество его обобщений.
          \end{itemize}
          \pause
    \item Математическая логика в целом
          \begin{itemize}
            \item Логики высших порядков.
            \item Изоморфизм Карри-Ховарда.
            \item Формализация математики.
            \item И автоматический поиск доказательств.
            \item Agda, Idris и Coq сюда же!
          \end{itemize}
          \pause
    \item Теория категорий: источник понятий функтора, монады, естественного преобразования.
  \end{itemize}
\end{frame}


\begin{frame}[fragile]
  \frametitle{Дополнительное чтение}
  \begin{itemize}
    \item \href{http://dev.stephendiehl.com/hask/}{What I Wish I Knew When Learning Haskell}
    \item \href{https://lexi-lambda.github.io/blog/2018/02/10/an-opinionated-guide-to-haskell-in-2018/}{An opinionated guide to Haskell in 2018}
    \item \href{https://realworldocaml.org/}{Real World OCaml: Functional Programming for the Masses} (книга, доступна бесплатно)
    \item \href{https://wiki.portal.chalmers.se/agda/pmwiki.php?n=Main.Documentation}{Много ссылок по Agda}
    \item \href{http://docs.idris-lang.org/en/latest/tutorial/index.html}{Введение в Idris}
    \item \href{https://doc.rust-lang.org/1.30.0/book/2018-edition/index.html}{The Rust Programming Language} (книга, доступна бесплатно)
    \item \href{https://github.com/hmemcpy/milewski-ctfp-pdf}{Category Theory for Programmers} (книга, доступна бесплатно)
    \item \href{https://www.hillelwayne.com/post/theorem-prover-showdown/}{The Great Theorem Prover Showdown}
    \item \href{https://blog.acolyer.org/2014/11/24/use-of-formal-methods-at-amazon-web-services/}{Use of Formal Methods at Amazon Web Services}
    \item \href{https://blog.acolyer.org/2015/01/12/how-to-write-a-21st-century-proof/}{How to write a 21st Century Proof}
  \end{itemize}
\end{frame}

\end{document}
