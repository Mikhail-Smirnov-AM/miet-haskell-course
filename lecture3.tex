\documentclass[10pt]{beamer}
\usepackage[utf8]{inputenc}
\usepackage[T1,T2A]{fontenc}
\usepackage[russian]{babel}
\usepackage{color}
\usepackage{calc}
\usepackage{graphicx}
\usepackage{epstopdf}
\usepackage{hyperref}
\hypersetup{unicode,colorlinks}
\usetheme[progressbar=head,numbering=fraction,block=fill]{metropolis}
\usepackage{minted}
\usepackage{dejavu}
%\usepackage{adjustbox}  % Позволяет сузить куски кода (или текст) ровно настолько, чтобы уместиться в слайд
\usepackage{csquotes}
\usepackage{upquote}

\usemintedstyle{solarized-light}
\newminted[haskell]{haskell}{
    escapeinside=!!,
    mathescape=true,
    texcomments=true,
    beameroverlays=true,
    autogobble=true,
    fontsize=\small,
    breaklines=false  % Лучше сам поставлю переносы на удобных местах
}
\newminted[haskellsmall]{haskell}{
    escapeinside=!!,
    mathescape=true,
    texcomments=true,
    beameroverlays=true,
    autogobble=true,
    fontsize=\footnotesize,
    breaklines=false
}
\newminted[haskelltiny]{haskell}{
    escapeinside=!!,
    mathescape=true,
    texcomments=true,
    beameroverlays=true,
    autogobble=true,
    fontsize=\scriptsize,
    breaklines=false
}
\newmintinline[haskinline]{haskell}{
    escapeinside=!!,
    mathescape=true,
    beameroverlays=true,
    breaklines=true
}
\newminted[ghci]{text}{
    autogobble=true,
    fontsize=\small,
    breaklines=false
}
\newminted[ghcismall]{text}{
    autogobble=true,
    fontsize=\footnotesize,
    breaklines=false
}
\newminted[ghcitiny]{text}{
    autogobble=true,
    fontsize=\scriptsize,
    breaklines=false
}
\newmintinline[ghcinline]{text}{
    breaklines=true
}

\newcommand{\hackage}[1]{\href{https://hackage.haskell.org/package/#1}{#1}}

\vfuzz=20pt  % позволяет тексту дойти до номера слайда

\author{Алексей Романов}
\subtitle{Функциональное программирование на Haskell}
%\logo{}
\institute{МИЭТ}
\subject{Функциональное программирование на Haskell}
%\setbeamercovered{transparent}
%\setbeamertemplate{navigation symbols}{}


\title{Лекция 3: типы данных}
\date{21 февраля 2018}

\begin{document}
\begin{frame}[plain]
\maketitle
\end{frame}

\begin{frame}[fragile]
\frametitle{Типы-\enquote{перечисления}}
\begin{itemize}
    \item Тип \lstinline|Bool| не встроен в язык, а определён в стандартной библиотеке:
\begin{lstlisting}
data Bool = False | True deriving (...)
\end{lstlisting}
    \item Ключевое слово \lstinline|data| начинает определение.
    \item \lstinline|Bool| "--- название типа.
    \item \lstinline|False| и \lstinline|True| называются конструкторами данных.
    \item Читаем как \enquote{У типа \lstinline|Bool| есть ровно два значения: \lstinline|False| и \lstinline|True|.} 
    \only<1> { \item О \lstinline|deriving| позже. }
    \pause
    \item Так определяется любой тип с фиксированным перечнем значений (как \lstinline|enum| в других языках):
\begin{lstlisting}
data Weekday = Monday | Tuesday | ...
data FileMode = Read | Write | Append | ...
\end{lstlisting}
    \item Какие образцы у этих типов (кроме переменных и \lstinline|_|)? \pause
    \item Каждое значение (конструктор) "--- образец.
\end{itemize}
\end{frame}

\begin{frame}[fragile]
\frametitle{Типы-\enquote{структуры}}
\begin{itemize}
    \item \lstinline|data| также используется для аналогов структур: типов с несколькими полями. Например,
\begin{lstlisting}[basicstyle=\ttfamily\footnotesize]
data Point = Point Double Double
\end{lstlisting}
    \item Первое \lstinline|Point| "--- название типа.
    \item Второе "--- конструктора.
    \item Когда конструктор один, названия обычно совпадают. Всегда по контексту можно определить, что из них имеется в виду.
    \item Образцы этого типа \pause это\\
\begin{lstlisting}[basicstyle=\ttfamily\footnotesize]
Point образец_Double образец_Double
\end{lstlisting}
    \item Зададим значение типа \lstinline|Point| и функцию на них:
\begin{lstlisting}[basicstyle=\ttfamily\footnotesize]
Prelude> let origin = Point 0 0

Prelude> let xCoord (Point x _) = x
Prelude> xCoord origin
0.0
\end{lstlisting}
\end{itemize}
\end{frame}

\begin{frame}[fragile]
\frametitle{Типы-\enquote{структуры}}
\begin{itemize}
    \item Конструктор является функцией
\begin{lstlisting}
Prelude> :t Point !\pause!
Point :: Double -> Double -> Point
\end{lstlisting}
    \item Мы можем дать полям конструктора имена. Это:\hypertarget{rec1}{}
    \begin{itemize}
        \item Документирует их смысл.
        \item Определяет функции, возвращающие их значения.
    \end{itemize}
\begin{lstlisting}
Prelude> data Point = Point { xCoord :: Double, yCoord :: Double } deriving Show
Prelude> :t yCoord
yCoord2 :: Point -> Double
Prelude> yCoord (Point 0 (-1))
-1.0
\end{lstlisting}
\item Особенно полезно, когда полей много, и только по типу не понять, какое где.
\item Если успеем, вернёмся к записям в конце лекции. \hyperlink{rec2}{\beamergotobutton{Синтаксис записей}}
\end{itemize}
\end{frame}

\begin{frame}[fragile]
\frametitle{Алгебраические типы: общий случай}
\begin{itemize}
    \item Может быть несколько конструкторов с полями:
\begin{lstlisting}[basicstyle=\ttfamily\small]
data IpAddress = IPv4 String | IPv6 String
\end{lstlisting}
\item Создание значения:
\begin{lstlisting}[basicstyle=\ttfamily\small]
Prelude> let googleDns = IPv4 "8.8.8.8"
\end{lstlisting}
\item Образцы: \lstinline[basicstyle=\ttfamily\small]|IPv4 образец_String| и \lstinline[basicstyle=\ttfamily\small]|IPv6 образец_String|.
\item Функции часто определяются по уравнению на каждый конструктор:
\begin{lstlisting}[basicstyle=\ttfamily\small]
asString (IPv4 ip4) = ip4
asString (IPv6 ip6) = ip6
\end{lstlisting}
\pause
\item Но не обязательно:
\begin{lstlisting}[basicstyle=\ttfamily\small]
isLocalhost (IPv4 "127.0.0.1") = True
isLocalhost (IPv6 "0:0:0:0:0:0:0:1") = True
isLocalhost _ = False
\end{lstlisting}
\pause
\item Ещё пример: \lstinline[basicstyle=\ttfamily\small]!data ImageFormat = JPG | PNG | IF String!
\end{itemize}
\end{frame}

\begin{frame}[fragile]
\frametitle{Типы с параметрами (полиморфные)}
\begin{itemize}
    \item В определении типа могут быть параметры.
    \item Пример:
\begin{lstlisting}
data Maybe a = Nothing | Just a
\end{lstlisting}
    \item \lstinline|Maybe| "--- конструктор типа.
    \item Вместо \lstinline|a| можно подставить любой тип, и получить снова тип: \lstinline|Maybe Int|, \lstinline|Maybe Bool|.
    \item Примеры их значений: \lstinline|Just 1|, \lstinline|Nothing|.
    \item В Haskell нет отношения \enquote{Надтип "--- подтип}, вместо него \enquote{Более общий "--- частный случай}.
    \item \lstinline|Maybe a| описывает необязательное значение типа \lstinline|a|.
    \pause
    \item Ещё часто встречающийся тип:
\begin{lstlisting}
data Either a b = Left a | Right b
\end{lstlisting}
    \item Могут быть сколь угодно сложные комбинации. 
    \item Например, \lstinline|Either (Either Int Char) (Maybe Bool)| со значением \lstinline|Right Nothing|.
\end{itemize}
\end{frame}

\begin{frame}[fragile]
\frametitle{\lstinline|Maybe| и \lstinline|null|}
\begin{itemize}
    \item \lstinline|Nothing| похоже на \lstinline|null| в C-подобных языках, но явно прописано в типах.
    \item Не нужно для каждой функции документировать, как работает с \lstinline|null|, достаточно посмотрить на наличие \lstinline|Maybe| в сигнатуре.
    \item И поэтому нет возможного несовпадения реализации с документацией.
    \item Нет типов, у которых \lstinline|null| нет и с отсутствием значения приходится что-то придумывать.
    \item Есть \lstinline|Maybe (Maybe a)| (бывает полезно).
    \pause
    \item \enquote{\itshape \lstinline|null| "--- моя ошибка ценой в миллиард долларов... Я планировал сделать любое использование ссылок [в Algol] абсолютно безопасным. Но не удержался от соблазна добавить \lstinline|null| просто потому, что его было так легко реализовать.} (\href{https://ru.wikipedia.org/wiki/%D0%A5%D0%BE%D0%B0%D1%80,_%D0%A7%D0%B0%D1%80%D0%BB%D1%8C%D0%B7_%D0%AD%D0%BD%D1%82%D0%BE%D0%BD%D0%B8_%D0%A0%D0%B8%D1%87%D0%B0%D1%80%D0%B4}{Тони Хоар}, \href{https://www.infoq.com/presentations/Null-References-The-Billion-Dollar-Mistake-Tony-Hoare}{QCon 2009})
\end{itemize}
\end{frame}

\begin{frame}[fragile]
\frametitle{Кортежи}
\begin{itemize}
    \item Мы могли бы определить типы 
\begin{lstlisting}
data Pair a b = Pair a b
data Triple a b = Triple a b
...
\end{lstlisting}
(заметьте разницу смыслов \lstinline|Pair|, \lstinline|a| и \lstinline|b| в левой и правой частях!)
    \item Это универсальные произведения.
    \pause
    \item Но определять их не нужно, так как такие типы уже есть, со специальным синтаксисом:\\
    \lstinline|(a, b)| (или \lstinline|(,) a b|),\\ \lstinline|(a, b, c)| (или \lstinline|(,,) a b c|) и т.д.
    \item Синтаксис значений и образцов для них аналогичный: 
    \lstinline|(True, Just 'a') :: |\pause\lstinline|(Bool, Maybe Char)|.
    \item Максимальный размер 62.
    \end{itemize}
\end{frame}

\begin{frame}[fragile]
\frametitle{Рекурсивные типы и списки}
\begin{itemize}
    \item Тип может быть рекурсивным, то есть в правой части используется тот тип, который мы определяем.
    \item Самый важный пример такого типа: списки.
    Мы могли бы их определить как 
\begin{lstlisting}
data List a = Nil | Cons a (List a)
\end{lstlisting}
Вместо этого они имеют специальный синтаксис, как и кортежи, как будто их определение
\begin{lstlisting}
data [a] = [] | a : [a]
\end{lstlisting}
    \item Любой \lstinline|[a]| или пуст, или имеет \emph{голову} типа \lstinline|a| и \emph{хвост} типа \lstinline|[a]|.
    \item \lstinline|:| правоассоциативно: \lstinline|1 : 2 : 4 : []| читается \lstinline|1 : (2 : (4 : []))|.
    \item Сокращение (относится и к значениям и к образцам):\\
    \lstinline|[x, y, z]| означает \lstinline|x : y : z : []|, а \lstinline|[x]| "--- \lstinline|x : []|.
\end{itemize}
\end{frame}

\begin{frame}[fragile]
\frametitle{Синонимы типов}
\begin{itemize}
    \item На этом с \lstinline|data| разобрались (пока что).
    \pause
    \item Есть ещё два вида определения типов.
    \item \lstinline|type| объявляет синоним типа "--- другое имя для существующего типа. 
    \item В стандартной библиотеке:
\begin{lstlisting}
type String = [Char]
type FilePath = String
\end{lstlisting}
    \item Оба определения теперь считаются плохими! Для \lstinline|String| подробности в будущих лекциях.
    \pause 
    \item Для \lstinline|FilePath| дело в том, что осмысленные операции над путями не те, что над строками:
\begin{lstlisting}
Prelude Data.List> isPrefixOf 
    ("C:\\Program Files" :: FilePath) 
    ("C:\\Program Files (x86)" :: FilePath)
True
\end{lstlisting}
\end{itemize}
\end{frame}

\begin{frame}[fragile]
\frametitle{\lstinline|newtype|}
\begin{itemize}
    \item \lstinline|newtype| как раз позволяет определить тип с тем же представлением, что существующий, но с другими операциями. Т.е. разумнее было бы 
\begin{lstlisting}
newtype FilePath = FilePath String
\end{lstlisting}
    \item Или \lstinline|FilePath [String]| со списком директорий.
    \item Ещё можно рассмотреть
\begin{lstlisting}
newtype EMail = EMail String
\end{lstlisting}
    \item У \lstinline|type| конструкторов значений нет, у \lstinline|newtype| "--- всегда один с одним полем.
    \item Между \lstinline|newtype| и \lstinline|data| с одним конструктором с одним полем есть разница из-за ленивости.
    \item Пока из них предпочитаем первое.
\end{itemize}
\end{frame}


\begin{frame}[fragile]
\frametitle{Полиморфные функции и значения}
\begin{itemize}
    \item Какой тип функции
\begin{lstlisting}
Prelude> let id x = x
Prelude> :t id !\pause!
id :: a -> a
\end{lstlisting}
\item Более явно: \lstinline|id :: forall a. a -> a|.
\begin{itemize}
\item Чтобы написать это в коде, нужно \lstinline|ExplicitForAll|.
\end{itemize}
\item Аналогично
\begin{lstlisting}
Prelude> let foo (_, x, y) = (y, x, x)
Prelude> :t foo !\pause!
id :: (a1, c, a2) -> (a2, c, c) 
-- или (a, b, c) -> (c, b, b)
\end{lstlisting}
\item \lstinline|Just :: a -> Maybe a|
\item И \lstinline|[] :: [a]| "--- полиморфные значения не обязательно функции.
\end{itemize}
\end{frame}

\begin{frame}[fragile]
\frametitle{Полезные расширения для полиморфных функций}
\begin{itemize}
    \item Параметр типа можно передать явно:
\begin{lstlisting}
Prelude> :set -XTypeApplications
Prelude> :t id @Int
id @Int :: Int -> Int
\end{lstlisting}
    \pause
    \item По умолчанию область видимости переменной типа "--- сигнатура, где она появилась. Например, в
\begin{lstlisting}
g :: [a] -> [a]
g (x:xs) = xs ++ [ x :: a ]
\end{lstlisting}
    \pause
    две переменные \lstinline|a| "--- разные и читаются как
\begin{lstlisting}
g :: forall a. [a] -> [a]
g (x:xs) = xs ++ [ x :: forall a. a ]
\end{lstlisting}
    \pause
    (так что функция не скомпилируется). 
\end{itemize}
\end{frame}

\begin{frame}[fragile]
\frametitle{Область видимости переменных типа}
\begin{itemize}
    \item Расширение \lstinline|ScopedTypeVariables| расширяет область видимости, но только для переменных с явным \lstinline|forall|.
    \item С ним пример с прошлого слайда можно записать как
\begin{lstlisting}
g :: forall a. [a] -> [a]
g (x:xs) = xs ++ [ x :: a ]
\end{lstlisting}
    \item Можно также вводить новые переменные сигнатурой для выражения \emph{внутри} определения, они будут видимы в этом выражении.
    \item Точные правила можно найти в документации, в разделе \href{http://downloads.haskell.org/~ghc/latest/docs/html/users_guide/glasgow_exts.html#lexically-scoped-type-variables}{Lexically scoped type variables}.
\end{itemize}
\end{frame}

\begin{frame}[fragile]
\frametitle{Структурно-рекурсивные функции}
\begin{itemize}
    \item Функции над рекурсивными типами часто тоже рекурсивны (или вызывают рекурсивные).
\begin{lstlisting}
length :: [a] -> Int
length [] = 0
length (_ : xs) = length xs + 1
\end{lstlisting}
    \item Натуральные числа "--- \enquote{почётный} рекурсивный тип.
    \pause
    \item Натуральное число это либо $0$, либо $n + 1$, где $n$ "--- натуральное число.
    \item Соответственно, функции над ними тоже определяются рекурсивно:
\begin{lstlisting}
replicate _ 0 = []
replicate x n = x : replicate x (n - 1)
\end{lstlisting}
\item Раньше \lstinline|n + 1| (и вообще \lstinline|+ константа|) было образцом, в Haskell 2010 их исключили.
\end{itemize}
\end{frame}

\begin{frame}[fragile]
\frametitle{Алгебраические типы}
\begin{itemize}
    \item Несколько полей у конструктора соответствуют операции декартова произведения в теории множеств.
    \onslide<3->{\begin{itemize}
            \item В логике конъюнкции, а в алгебре произведению.
        \end{itemize}}
    \item А несколько конструкторов\onslide<1>{?}\onslide<2->{операции дизъюнктного объединения $A_1 \sqcup A_2 \sqcup \ldots = \{1\} \times A_1 \cup \{2\} \times A_2 \cup \ldots$}
    \onslide<3->{
        \begin{itemize}
        \item В логике дизъюнкции, а в алгебре сумме.
        \end{itemize}
    }
    \onslide<2->{
    \item Сравните помеченное объединение с непомеченным \lstinline|union| в C/C++.
    \item И с \lstinline|boost::variant| (\lstinline|std::variant| в C++17).
    }
    \onslide<3->{\item Тип функций "--- множество функций в теории множеств, импликация в логике и возведение в степень в алгебре.}
    \onslide<4->{\item Обычные законы алгебры выполняются для типов с точностью до изоморфизма, как и для множеств.}
    % TODO переделать в таблицу
\end{itemize}
\end{frame}

\begin{frame}[fragile]
\frametitle{Типы и модули}
\begin{itemize}
    \item Типы можно импортировать и экспортировать.
    \item \lstinline|НазваниеТипа(..)| в списке экспорта указывает, что конструкторы тоже включены в список.
    \item Просто \lstinline|НазваниеТипа| экспортирует только тип без конструкторов.
    \item Можно также перечислить явно, какие именно конструкторы экспортируются.
    \item Для импорта всё аналогично.
\end{itemize}
\end{frame}

\begin{frame}[fragile]
\frametitle{Записи}\hypertarget{rec2}{}
\hyperlink{rec1}{\beamerreturnbutton{К определению записей}}
\begin{itemize}
\item 
\begin{lstlisting}[basicstyle=\ttfamily\small]
Prelude> let aPoint = Point 0 (-1)
Prelude> aPoint { xCoord = 1 } !\pause!
Point {xCoord = 1.0, yCoord = -1.0}
Prelude> aPoint !\pause!
Point {xCoord = 0.0, yCoord = -1.0} !\pause!
Prelude> let Point {xCoord = x'} = it
Prelude> x' !\pause!
0.0 !\pause!

Prelude> :set -XNamedFieldPuns -XRecordWildCards
Prelude> let f (Point { xCoord, yCoord }) = xCoord + yCoord
Prelude> let g (Point { .. }) = x + y
\end{lstlisting}
\item Два последних образца "--- сокращения\\ \lstinline[basicstyle=\ttfamily\small]|Point { xCoord = xCoord, yCoord = yCoord }|.
\end{itemize}

\end{frame}

\begin{frame}[fragile]
\frametitle{Записи: минусы}\hypertarget{rec2}{}
\hyperlink{rec1}{\beamerreturnbutton{К определению записей}}
\begin{itemize}
    \item
\begin{lstlisting}
Prelude> Point { xCoord = 0 }
\end{lstlisting}
    \pause
    без \lstinline|yCoord| компилируется только с предупреждением (можно сделать ошибкой с помощью опций компилятора).
    \pause 
    \item Записи можно применять для типов с несколькими конструкторами, но это создаёт частичные (не всюду определённые) функции.
    \pause
    \item Одинаковые названия полей у двух записей ведут к проблемам. 
    \pause
    \item Получается две функции с одинаковыми названиями.
    \item \lstinline|DisambiguateRecordFields| и \lstinline|DuplicateRecordFields| улучшают ситуацию.
    \pause
    \item Есть ещё решения на уровне библиотек.
\end{itemize}
\end{frame}

\end{document}