\documentclass[11pt]{beamer}
\usepackage[utf8]{inputenc}
\usepackage[T1,T2A]{fontenc}
\usepackage[russian]{babel}
\usepackage{color}
\usepackage{calc}
\usepackage{graphicx}
\usepackage{epstopdf}
\usepackage{hyperref}
\hypersetup{unicode,colorlinks}
\usetheme[progressbar=head,numbering=fraction,block=fill]{metropolis}
\usepackage{minted}
\usepackage{dejavu}
%\usepackage{adjustbox}  % Позволяет сузить куски кода (или текст) ровно настолько, чтобы уместиться в слайд
\usepackage{csquotes}
\usepackage{upquote}

\usemintedstyle{solarized-light}
\newminted[haskell]{haskell}{
    escapeinside=!!,
    mathescape=true,
    texcomments=true,
    beameroverlays=true,
    autogobble=true,
    fontsize=\small,
    breaklines=false  % Лучше сам поставлю переносы на удобных местах
}
\newminted[haskellsmall]{haskell}{
    escapeinside=!!,
    mathescape=true,
    texcomments=true,
    beameroverlays=true,
    autogobble=true,
    fontsize=\footnotesize,
    breaklines=false
}
\newminted[haskelltiny]{haskell}{
    escapeinside=!!,
    mathescape=true,
    texcomments=true,
    beameroverlays=true,
    autogobble=true,
    fontsize=\scriptsize,
    breaklines=false
}
\newmintinline[haskinline]{haskell}{
    escapeinside=!!,
    mathescape=true,
    beameroverlays=true,
    breaklines=true
}
\newminted[ghci]{text}{
    autogobble=true,
    fontsize=\small,
    breaklines=false
}
\newminted[ghcismall]{text}{
    autogobble=true,
    fontsize=\footnotesize,
    breaklines=false
}
\newminted[ghcitiny]{text}{
    autogobble=true,
    fontsize=\scriptsize,
    breaklines=false
}
\newmintinline[ghcinline]{text}{
    breaklines=true
}

\newcommand{\hackage}[1]{\href{https://hackage.haskell.org/package/#1}{#1}}

\vfuzz=20pt  % позволяет тексту дойти до номера слайда

\author{Алексей Романов}
\subtitle{Функциональное программирование на Haskell}
%\logo{}
\institute{МИЭТ}
\subject{Функциональное программирование на Haskell}
%\setbeamercovered{transparent}
%\setbeamertemplate{navigation symbols}{}

\usepackage{syntax}

\title{Лекция 12: тестирование свойств}

\begin{document}
\begin{frame}[plain]
  \maketitle
\end{frame}

\begin{frame}[fragile]
  \frametitle{Привычный подход к тестированию}
  \begin{itemize}
    \item Как обычно мы тестируем написанные функции?
          \pause
    \item Вычисляем функцию на каких-то конкретных аргументах и сравниваем результат с ожидаемым.
    \item Плюсы:
          \pause
          \begin{itemize}
            \item Тесты легко понять.
            \item Можно сделать даже без помощи библиотек.
            \item Но при этом библиотеки для написания таких тестов есть практически для всех языков.
            \item Для чистых функций (в Haskell почти для всех!) кроме аргументов и результата функции проверять и нечего.
          \end{itemize}
          \pause
    \item TDD (Test Driven Development) \pause "--- сначала тесты, потом реализация.
  \end{itemize}
\end{frame}

\begin{frame}[fragile]
  \frametitle{Недостатки этого подхода}
  \begin{itemize}
    \item А какие у этого подхода минусы?
          \pause
    \item Сколько нужно тестов? Всегда можно добавить ещё.
          \pause
          Разве что если функция на \lstinline|Bool|...
          \pause
    \item Нужно искать пограничные случаи (т.е. те, в которых ошибка более вероятна) самому. \\
          Хотя есть часто встречающиеся: \pause \lstinline|0|, пустой список, \lstinline|INT_MIN|...
          \pause
    \item Для каких-то аргументов мы можем не знать правильного результата (и не иметь независимого от тестируемой функции способа его найти).
          \pause
    \item Есть ситуации, когда программу или функцию в ней намеренно пытаются сломать. \pause Например, злобный преподаватель.
  \end{itemize}
\end{frame}

\begin{frame}[fragile]
  \frametitle{Свойства вместо примеров}
  \begin{itemize}
    \item Тестирование свойств (property testing или property-based testing) позволяет избежать этих проблем (и создать новые, как обычно).
    \item Основная идея: вместо результата на конкретных аргументах мы описываем свойства, которые должны выполняться \emph{для всех аргументов} подходящего типа. \pause Или не совсем для всех, но об этом потом.
          \pause
    \item Эти свойства проверяем на случайных аргументах (скажем, 100 за раз по умолчанию).
          \pause
    \item В процессе разработки тесты запускаются далеко не один раз, так что каждое свойство будет проверено в сотнях или тысячах случаев.
  \end{itemize}
\end{frame}

% TODO: 1. Перейти на Hedgehog; 
% TODO: 2. Добавить примеры неверных свойств без исключений!

\begin{frame}[fragile]
  \frametitle{Простейший пример}
  \begin{itemize}
    \item Для Haskell две основных библиотеки: \href{http://hackage.haskell.org/package/QuickCheck}{QuickCheck} и \href{https://github.com/hedgehogqa/haskell-hedgehog/}{Hedgehog}. Мы используем первую для простоты кода.
          \begin{lstlisting}[basicstyle=\ttfamily\small]
> import Test.QuickCheck
> quickCheck (\x -> (x :: Integer) * x >= 0)
+++ OK, passed 100 tests.
\end{lstlisting}
    \item Проверили, что квадраты \lstinline|Integer| неотрицательны.
  \end{itemize}
\end{frame}

\begin{frame}[fragile]
  \frametitle{Ещё один пример}
  \begin{itemize}
    \item Верно ли, что первый элемент списка всегда равен последнему элементу развёрнутого?
          \begin{lstlisting}[escapeinside=||,basicstyle=\ttfamily\small]
> quickCheck (\xs -> head xs === last (reverse xs)) |\pause|
*** Failed! Exception: 'Prelude.head: empty list' (after 1 test):
[]
\end{lstlisting}
    \item Как исправить? \pause Нужно добавить условие, что список непустой:
          \begin{lstlisting}[escapeinside=||,basicstyle=\ttfamily\small]
> quickCheck (\xs -> not (null xs) ==> head xs === last (reverse xs)) |\pause|
+++ OK, passed 100 tests.
\end{lstlisting}

  \end{itemize}
\end{frame}

\begin{frame}[fragile]
  \frametitle{Основные типы и функции}
  \begin{itemize}
    \item \lstinline|Gen a|: генератор случайных значений типа \lstinline|a|.
    \item \lstinline|forAll :: (Show a, Testable prop) => Gen a -> (a -> prop) -> Property|: создаёт свойство с использованием заданного генератора.
    \item \lstinline|class Arbitrary a where arbitrary :: Gen a|: типы, у которых есть \enquote{стандартный генератор}.
    \item[]
    \item \lstinline|class Testable a|: типы, которые можно протестировать (например \lstinline|Bool| и \lstinline|Property|).
    \item Функции \lstinline|a -> prop|, где \lstinline|Arbitrary a|, \lstinline|Show a|, \lstinline|Testable prop| "--- тоже \lstinline|Testable|.
    \item \lstinline|quickCheck, verboseCheck :: Testable prop => prop -> IO ()|: проверяет свойство и выводит результат.
  \end{itemize}
\end{frame}

\begin{frame}[fragile]
  \frametitle{Генераторы значений}
  \begin{itemize}
    \item Примеры базовых генераторов:
          \begin{lstlisting}[basicstyle=\ttfamily\small]
choose :: Random a => (a, a) -> Gen a
elements :: [a] -> Gen a
\end{lstlisting}
    \item Из простых генераторов можно строить более сложные, пользуясь тем, что \lstinline|Gen| "--- монада. Например:
          \begin{lstlisting}[basicstyle=\ttfamily\small]
genPair :: Gen a -> Gen b -> Gen (a, b)
genPair ga gb =!\pause! liftA2 (,) ga gb

genMaybe :: Gen a -> Gen (Maybe a)
genMaybe ga = do is_just <- arbitrary @Bool
                 if is_just 
                   then!\pause! fmap Just ga
                   else!\pause! pure Nothing
\end{lstlisting}
  \end{itemize}
\end{frame}

\begin{frame}[fragile]
  \frametitle{Генераторы значений}
  \begin{itemize}
    \item С рекурсивными типами несколько сложнее. Можно попробовать привести пример
          \begin{lstlisting}[basicstyle=\ttfamily\scriptsize]
data Tree a = Empty | Node a (Tree a) (Tree a)

treeOf :: Gen a -> Gen (Tree a)
treeOf g = oneOf [pure Empty, 
                  liftA3 Node g (treeOf g) (treeOf g)]
\end{lstlisting}
          но это определение плохое: оно даёт очень маленькие деревья (почему?).
    \item Правильнее:
          \begin{lstlisting}[basicstyle=\ttfamily\scriptsize]
treeOf g = do n <- getSize
              if n == 0
                then pure Empty
                else do
                  k <- choose (0,n)
                  liftA3 Node g (resize k $ treeOf g)
                                (resize (n-k) $ treeOf g)
\end{lstlisting}
  \end{itemize}
\end{frame}

\begin{frame}[fragile]
  \frametitle{Минимизация контрпримеров}
  \begin{itemize}
    \item Найденный сначала контрпример обычно будет довольно большим (хотя начинается генерация с примеров малого размера).
    \item Поэтому библиотека старается найти и выдать пользователю более простые значения аргументов, дающие ту же ошибку.
    \item В случае QuickCheck за это уменьшение отвечает функция \lstinline|shrink :: a -> [a]| в классе \lstinline|Arbitrary|.
    \item В Hedgehog, это уменьшение задаётся как часть \lstinline|Hedgehog.Gen|, что имеет преимущества:
          \begin{itemize}
            \item Если что-то сгенерировали, точно можно уменьшить.
            \item При уменьшении сохраняются инварианты.
            \item Подробнее: \href{https://hypothesis.works/articles/integrated-shrinking/}{Integrated vs type based shrinking}.
          \end{itemize}
  \end{itemize}
\end{frame}

\begin{frame}[fragile]
  \frametitle{Воспроизводимость}
  \begin{itemize}
    \item Если мы нашли контрпример (например, с десятого запуска) и изменили реализацию, нужно проверить, что этот случай исправился.
    \item В QuickCheck для этого просто запускаем то же свойство с напечатанными аргументами. Можно также добавить обычный тест с ними.
    \item Конечно, это удобно делать только тогда, когда минимизация более-менее удалась.
    \item Кроме того, для некоторых типов результат \lstinline|show x| недостаточен для точного воспроизведения \lstinline|x| (придумаете пример?)
    \item Поэтому в Hedgehog кроме аргументов выдаётся параметр для перезапуска генератора.
  \end{itemize}
\end{frame}

\begin{frame}[fragile]
  \frametitle{Реализация Arbitrary для своих типов}
  \begin{itemize}
    \item Для пар:
          \begin{lstlisting}[basicstyle=\ttfamily\small]
instance (Arbitrary a, Arbitrary b) => Arbitrary (a, b) where
  arbitrary = genPair arbitrary arbitrary
  shrink (x, y) = [(x', y') | x' <- shrink x, y' <- shrink y]
\end{lstlisting}
    \item Для деревьев:
          \begin{lstlisting}[basicstyle=\ttfamily\small]
instance Arbitrary a => Arbitrary (Tree a) where
  arbitrary = treeOf arbitrary
  shrink Empty = []
  shrink (Node x l r) = 
    [Empty, l, r] ++
    [Node x' l' r' | 
      (x', l', r') <- shrink (x, l, r)]
\end{lstlisting}
  \end{itemize}
\end{frame}

\begin{frame}[fragile]
  \frametitle{Часто встречающиеся виды свойств}
  \begin{itemize}
    \item Самое тривиальное свойство "--- функция даёт результат, а не выкидывает исключение.
    \item Сверка с другой реализацией той же функции.
    \item При наличии также обратной функции: \lstinline|encode (decode x) == x|.
    \item Алгебраические свойства: идемпотентность, коммутативность, ассоциативность и так далее.
    \item Законы классов типов.
  \end{itemize}
\end{frame}

\begin{frame}[fragile]
  \frametitle{Генераторы функций}
  \begin{itemize}
    \item Для тестирования функций высшего порядка нужно уметь генерировать их аргументы, то есть случайные функции. Для этого в QuickCheck есть \href{http://hackage.haskell.org/package/QuickCheck-2.13.1/docs/Test-QuickCheck.html#g:14}{тип \lstinline|Fun| и класс \lstinline|Function|}. \lstinline|Fun a b| представляет функцию \lstinline|a -> b| как набор пар аргумент-значение и значение по умолчанию (например \lstinline|{"elephant"->1, "monkey"->1, _->0}|). Экземпляр \lstinline|Function a| нужен для генерации \lstinline|Fun a b|.
  \end{itemize}
\end{frame}

\begin{frame}[fragile]
  \frametitle{Свойства функций}
  \begin{itemize}
    \item Как пример использования, покажем, что функции \lstinline|Int -> String| "--- чистые, то есть дают одинаковый результат при повторном вызове:
          \begin{lstlisting}
prop :: Fun Int String -> Int -> Bool
prop (Fn f) x = f x == f x
> quickCheck prop
+++ OK, passed 100 tests.
\end{lstlisting}
  \end{itemize}
\end{frame}

\begin{frame}[fragile]
  \frametitle{Тестирование законов классов}
  \begin{itemize}
    \item Библиотека \href{http://hackage.haskell.org/package/quickcheck-classes}{quickcheck-classes} содержит свойства, проверяющие законы для множества классов.
    \item Например, законы \lstinline|Eq| определяются функцией \lstinline|eqLaws :: (Eq a, Arbitrary a, Show a) => Proxy a -> Laws| и могут быть проверены для \lstinline|Rational| с помощью \lstinline|lawsCheck (eqLaws (Proxy @Rational)))|.
    \item Для проверки законов \lstinline|Functor/Applicative/Monad/...| вместо \lstinline|Proxy| нужно \lstinline|Proxy1|.
    \item Для Hedgehog совершенно аналогичная \href{https://hackage.haskell.org/package/hedgehog-classes}{hedgehog-classes} (только последняя версия у меня не поставилась).
  \end{itemize}
\end{frame}

\begin{frame}[fragile]
  \frametitle{Ещё раз: Hedgehog и QuickCheck}
  \begin{itemize}
    \item Главная разница: в Hedgehog уменьшение интегрировано с генерацией значений, а не определяется типом.
    \item Нет \lstinline|Arbitrary|, генераторы контролируются полностью явно.
    \item Для улучшения воспроизводимости вместе с контрпримером выдаётся seed.
    \item Улучшенный показ контрпримеров с помощью пакета \href{https://hackage.haskell.org/package/pretty-show}{\lstinline|pretty-show|} и функции \href{http://hackage.haskell.org/package/hedgehog-1.0/docs/Hedgehog.html#v:diff}{\lstinline|diff|}.
    \item Проще работа с монадическими свойствами.
    \item Поддержка тестирования с состоянием.
    \item Для генерации функций нужна отдельная библиотека \href{https://hackage.haskell.org/package/hedgehog-fn}{\lstinline|hedgehog-fn|}.
    \item Параллельная проверка (для ускорения).
  \end{itemize}
\end{frame}

\begin{frame}[fragile]
  \frametitle{Дополнительное чтение}
  \begin{itemize}
    \item \href{https://hypothesis.works/articles/quickcheck-in-every-language/}{QuickCheck in Every Language} Список библиотек тестирования свойств на разных языках на апрель 2016 (от автора Hypothesis для Python).
    \item \href{https://begriffs.com/posts/2017-01-14-design-use-quickcheck.html}{The Design and Use of QuickCheck}
    \item \href{https://teh.id.au/posts/2017/04/23/property-testing-with-hedgehog/}{Property testing with Hedgehog}
    \item \href{http://www.well-typed.com/blog/2019/05/integrated-shrinking/}{Integrated versus Manual Shrinking} Реализация и сравнение мини-версий QuickCheck и Hedgehog.
    \item \href{https://habr.com/ru/post/434008/}{Как перестать беспокоиться и начать писать тесты на основе свойств}
    \item \href{https://www.hillelwayne.com/post/contract-examples/}{Finding Property Tests}
    \item \href{https://wickstrom.tech/programming/2019/03/02/property-based-testing-in-a-screencast-editor-introduction.html}{Property-Based Testing in a Screencast Editor}
    \item \href{https://austinrochford.com/posts/2014-05-27-quickcheck-laws.html}{Verifying Typeclass Laws in Haskell with QuickCheck}
    \item \href{https://blog.jrheard.com/hypothesis-and-pexpect}{Using Hypothesis and Pexpect to Test High School Programming Assignments}
  \end{itemize}
\end{frame}

\end{document}
